%\documentclass[11pt]{beamer}
\documentclass[11pt,handout,aspectratio=1610]{beamer}

\usepackage[utf8]{inputenc}
\usepackage[T1]{fontenc}
\usepackage[spanish]{babel}
\usepackage{latexsym} 
\usepackage{amsmath}
\usepackage{amsfonts}
\usepackage{amssymb}
\usepackage{esint}
\usepackage{array}
\usepackage{multirow}
\usepackage{xcolor}
\usepackage{graphicx}
\usepackage{tikz}
\usepackage{tikz-3dplot}
\usetikzlibrary{babel}
\usetikzlibrary{calc,patterns,decorations.pathmorphing,decorations.markings}
\usepackage{xcolor}
\usepackage{epstopdf}
\usepackage[nointegrals]{wasysym}
\usepackage{hyperref}
\usepackage{cancel}
\usepackage[font=small,labelfont={small,bf},margin=0.5cm,justification=justified]{caption}
\usepackage[font=small,labelfont={small,bf}]{subcaption}

\usetheme{Berkeley}
\usecolortheme{seahorse}
\uselanguage{Spanish}

\newcommand{\sgn}{\mathop{\text{sgn}}}
\newcommand{\diff}[0]{\text{d}}
\newcommand{\fdiff}[2]{\dfrac{\text{d} #1}{\text{d} #2}}
\newcommand{\pdiff}[2]{\frac{\partial #1}{\partial #2}}
\newcommand{\fddiff}[2]{\frac{\text{d^2} #1}{\text{d} #2^2}}
\newcommand{\grado}[0]{^{\circ}}
\newcommand{\chel}[4]{^{#1}_{#2}\text{#3}^{#4}}
\newcommand{\valmed}[1]{\left\langle #1 \right\rangle}
\newcommand{\E}[1]{\times 10^{#1}}
\newcommand{\ver}[1]{\hat{\vec{#1}}}
\newcommand{\vecg}[1]{\boldsymbol{#1}}
\newcommand{\iu}{\text{i}}
\newcommand{\norm}[1]{\left\vert\left\vert #1 \right\vert\right\vert}
\newcommand{\abs}[1]{\left\vert #1 \right\vert}
\newcommand{\tens}[1]{\mathbb{#1}}
\newcommand{\rr}{\mathbb{R}}
\newcommand{\logoUNAHUR}{\includegraphics[scale=0.15]{/home/shluna/Proyectos/Clases_Fisica_III/imgs/logo-universidad-nacional-de-hurlingham_preview_rev_1.png}}
\newcommand{\vs}{\vspace{11pt}}
\newcommand{\un}[1]{\text{#1}}

\title{Fenómenos ondulatorios}
\subtitle{Unidad 4}
\author{Física III}
\institute{Instituto de Tecnología e Ingeniería \\ \vspace{0.25cm} Universidad Nacional de Hurlingham}
\date{ }
\logo{\logoUNAHUR}

\AtBeginSection[]{
  \begin{frame}
  \vfill
  \centering
  \begin{beamercolorbox}[sep=8pt,center,shadow=true,rounded=true]{title}
    \usebeamerfont{title}\insertsectionhead\par%
  \end{beamercolorbox}
  \vfill
  \end{frame}
}

\tdplotsetmaincoords{70}{110}

\begin{document}

\frame{\titlepage}

\begin{frame}{En esta clase veremos:}
    \tableofcontents
\end{frame}

\section{Introducción}

\begin{frame}{Introducción}

    Una de las características más interesantes de los medios capaces de deformarse es la de transmitir ondas de un punto a otro dentro de su extensión, tal como ocurre cuando se arroja una pequeña piedra a una masa de agua estancada.

    \vs

    La presente unidad tiene como propósito estudiar el fenómeno de la transmisión de las ondas en el espacio, pero para ello debemos definir algunos conceptos.

    \begin{block}{Definición}
        Una \emph{onda} es una perturbación o señal, generada por un \emph{emisor} que se propaga a través de un \emph{medio} y, eventualmente, llega hasta un \emph{receptor}.  
    \end{block}

    Podemos notar en esta definición que para generar, transmitir y detectar una onda son necesarios un emisor, un medio y un receptor, respectivamente.

\end{frame}

\begin{frame}
    \frametitle{Tipos de onda}

    Las ondas pueden clasificarse según el medio en el que se propaguen. Tres ejemplos importantes son las siguientes:
    \begin{itemize}
        \item \textbf{Ondas mecánicas}: Son aquellas que se propagan a través de un medio material, como por ejemplo en un sólido deformable (\emph{ondas elásticas}) y en un fluido (\emph{olas}, \emph{ondas de presión}, \emph{sonido}).
        \item \textbf{Ondas elétromagnéticas}: Como su nombre lo sugiere, son las que se propagan a través del campo electromagnético, tales como la luz visible, ultravioleta, rayos X, rayos gamma, infrarrojo, etc.
        \item \textbf{Ondas gravitacionales}: Aquellas que se propagan a través del campo gravitatorio.
    \end{itemize}

\end{frame}

\section{Descripción matemática de las ondas}

\begin{frame}{Descripción matemática de las ondas}

    
    
\end{frame}

\section{La ecuación de ondas clásica}

\begin{frame}{La ecuación de ondas clásica}

    La ecuación de ondas clásica, o de D'Alembert, es una de las más importantes de la Física dado que describe \emph{cualquier} tipo de onda clásica, independientemente del medio y de la manera en que se propague.

    \vs

    Para derivarla, vamos a considerar una cuerda tensa capaz de deformarse.

\end{frame}

\end{document}

\begin{frame}{}



\end{frame}