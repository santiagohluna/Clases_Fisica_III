%\documentclass[11pt]{beamer}
\documentclass[11pt,handout,aspectratio=1610]{beamer}

\usepackage[utf8]{inputenc}
\usepackage[T1]{fontenc}
\usepackage[spanish]{babel}
\usepackage{latexsym} 
\usepackage{amsmath}
\usepackage{amsfonts}
\usepackage{amssymb}
\usepackage{esint}
\usepackage{array}
\usepackage{multirow}
\usepackage{xcolor}
\usepackage{graphicx}
\usepackage{tikz}
\usepackage{tikz-3dplot}
\usetikzlibrary{babel}
\usetikzlibrary{calc,patterns,decorations.pathmorphing,decorations.markings}
\usetikzlibrary{arrows.meta} % for arrow size
\tikzset{>=latex} % for LaTeX arrow head
\usepackage{xcolor}
\usepackage{epstopdf}
\usepackage[nointegrals]{wasysym}
\usepackage{hyperref}
\usepackage{cancel}
\usepackage[font=small,labelfont={small,bf},margin=0.5cm,justification=justified]{caption}
\usepackage[font=small,labelfont={small,bf}]{subcaption}

\usetheme{Berkeley}
\usecolortheme{seahorse}
\uselanguage{Spanish}

\newcommand{\sgn}{\mathop{\text{sgn}}}
\newcommand{\diff}[0]{\text{d}}
\newcommand{\fdiff}[2]{\dfrac{\text{d} #1}{\text{d} #2}}
\newcommand{\pdiff}[2]{\frac{\partial #1}{\partial #2}}
\newcommand{\pddiff}[2]{\frac{\partial^2 #1}{\partial #2^2}}
\newcommand{\fddiff}[2]{\frac{\diff^2 #1}{\diff #2^2}}
\newcommand{\grado}[0]{^{\circ}}
\newcommand{\chel}[4]{^{#1}_{#2}\text{#3}^{#4}}
\newcommand{\valmed}[1]{\left\langle #1 \right\rangle}
\newcommand{\E}[1]{\times 10^{#1}}
\newcommand{\ver}[1]{\hat{\mathbf{#1}}}
\newcommand{\vecg}[1]{\boldsymbol{#1}}
\newcommand{\iu}{\text{i}}
\newcommand{\norm}[1]{\left\vert\left\vert #1 \right\vert\right\vert}
\newcommand{\abs}[1]{\left\vert #1 \right\vert}
\newcommand{\tens}[1]{\mathbb{#1}}
\newcommand{\rr}{\mathbb{R}}
\newcommand{\logoUNAHUR}{\includegraphics[scale=0.15]{/home/shluna/Proyectos/Clases_Fisica_III/imgs/logo-universidad-nacional-de-hurlingham_preview_rev_1.png}}
\newcommand{\vs}{\vspace{11pt}}
\newcommand{\un}[1]{\text{#1}}

\title{Fenómenos ondulatorios}
\subtitle{Unidad 4}
\author{Física III}
\institute{Instituto de Tecnología e Ingeniería \\ \vspace{0.25cm} Universidad Nacional de Hurlingham}
\date{ }
\logo{\logoUNAHUR}

\AtBeginSection[]{
  \begin{frame}
  \vfill
  \centering
  \begin{beamercolorbox}[sep=8pt,center,shadow=true,rounded=true]{title}
    \usebeamerfont{title}\insertsectionhead\par%
  \end{beamercolorbox}
  \vfill
  \end{frame}
}

\tdplotsetmaincoords{70}{110}

\begin{document}

\frame{\titlepage}

\begin{frame}{En esta clase veremos:}
    \tableofcontents
\end{frame}

\section{Introducción}

\begin{frame}{Introducción}

    Una de las características más interesantes de los medios capaces de deformarse es la de transmitir ondas de un punto a otro dentro de su extensión, tal como ocurre cuando se arroja una pequeña piedra a una masa de agua estancada.

    \vs

    La presente unidad tiene como propósito estudiar el fenómeno de la transmisión de las ondas en el espacio, pero para ello debemos definir algunos conceptos.

    \begin{block}{Definición}
        Una \emph{onda} es una perturbación o señal, generada por un \emph{emisor} que se propaga a través de un \emph{medio} y, eventualmente, llega hasta un \emph{receptor}.  
    \end{block}

    Podemos notar en esta definición que para generar, transmitir y detectar una onda son necesarios un emisor, un medio y un receptor, respectivamente.

\end{frame}

\begin{frame}
    \frametitle{Tipos de onda}

    Las ondas pueden clasificarse según el medio en el que se propaguen. Tres ejemplos importantes son las siguientes:
    \begin{itemize}
        \item \textbf{Ondas mecánicas}: Son aquellas que se propagan a través de un medio material, como por ejemplo en un sólido deformable (\emph{ondas elásticas}) y en un fluido (\emph{olas}, \emph{ondas de presión}, \emph{sonido}).
        \item \textbf{Ondas elétromagnéticas}: Como su nombre lo sugiere, son las que se propagan a través del campo electromagnético, tales como la luz visible, ultravioleta, rayos X, rayos gamma, infrarrojo, etc.
        \item \textbf{Ondas gravitacionales}: Aquellas que se propagan a través del campo gravitatorio.
    \end{itemize}

\end{frame}

\section{Descripción matemática de las ondas}

\begin{frame}{Descripción matemática de las ondas}

    Anteriormente señalamos que una onda es una perturbación que se propaga por un determinado medio. Entendemos por perturbación a toda modificación o apartamiento del estado de equilibro, por lo tanto, es de fundamental importancia identificar las variables que definen el estado de equilibrio de un sistema dado. A modo de ejemplo, consideremos una cuerda tensa que se encuentra de forma horizontal sujeta a dos soportes, tal como se muestra en la Figura~\ref{fig:cuerda_equilibrio}.

    \vs 
    \begin{figure}
        \centering
        \includegraphics[width=0.6\textwidth]{../figs/cuerda_equilibrio.pdf}
        \caption{Cuerda tensa en equilibrio.}
        \label{fig:cuerda_equilibrio}
    \end{figure}

    Supongamos ahora que en el instante $t_0$ se le da a la cuerda un impulso tal que la cuerda se genera una perturbación que se propaga hacia la derecha con cierta rapidez $v$, como la que se muestra a continuación.

\end{frame}

\begin{frame}{Descripción matemática de las ondas}

    Supongamos ahora que en cierto instante se le da a la cuerda un impulso tal se genera una perturbación que se propaga hacia la derecha con cierta rapidez $v$, como la que se muestra a continuación.

    \begin{figure}
        \centering
        \begin{tikzpicture}[scale=1]
            \def\Px{1.8}

            \draw[->,thick] (-0.5,0) -- (5,0) node[below]{$x$};
            \draw[->,thick] (0,-0.5) -- (0,2) node[left]{$y$};
            \draw[blue,very thick,samples=100,smooth,variable=\x,domain=0:4.5] plot(\x,{1.25*exp(-(\x-2)^2/0.5)});

            \draw[thick,red,->] (2,1.5) -- node[above,midway]{$\vec{v}$} (3,1.5); 
            
            \coordinate (P) at (\Px,{1.25*exp(-(\Px-2)^2/0.5)});
            \coordinate (Px) at (\Px,0);
            
            \fill[black] (P) circle (0.5mm);
            \fill[black] (Px) circle (0.5mm);
            
            \draw[<->] (Px) -- node[right,midway]{$y$} (P);
        \end{tikzpicture}
    \end{figure}

    En este caso, la perturbación se describe como la distancia $y$ (hacia arriba o hacia abajo) en la que la cuerda se aparta de la línea horizontal que caracteriza su estado de equilibrio.

    \vs

    La coordenada $y$ de cada punto de la cuerda debe ser una función tanto de su posición $x$ como del tiempo $t$, esto es: $y = f(x,t)$. 

\end{frame}

\begin{frame}{Descripción matemática de las ondas}

    Puede comprobarse que, en la medida que puedan despreciarse todos los efectos disipativos, la forma del pulso se conserva a lo largo de todo su recorrido por la cuerda y que este se propaga a velocidad constante.

    \vs 

    Si $y = f(x,t_0)$ representa la forma del pulso en el instante $t_0$, en un instante de tiempo arbitrario $t$ posterior o anterior, el pulso se desplaza $s$ unidades hacia la derecha o hacia la izquierda, respectivamente, donde $$ s(t) = x_0 + v \left(t - t_0\right), $$ donde $x_0$ es el valor de $s$ en $t_0$, y, por lo tanto, en el instante $t$ la forma del pulso está dada por $y = f(x-s(t))$. En virtud de que la dependencia temporal está incluida en la expresión de $s(t)$, no es necesario incluirla en la forma $f(x-s(t),t)$.
 
\end{frame}

\begin{frame}{Descripción matemática de las ondas}
    
    Por otro lado, si la forma del pulso ha de conservarse, entonces la altura $y$ de un punto cuya abscisa es $x_1$, en cierto instante $t_1$, al cabo de cierto intervalo de tiempo $\Delta t$, la altura de otro punto de abscisa $x_2 = x_1 + v \, \Delta t$ debe ser también $y$ en el instante $t_2 = t_1 + \Delta t$, esto es: $$ y(x_2,t_2) = y(x_1,t_1) $$  

    \begin{figure}
        \centering
        \begin{tikzpicture}[scale=1]
            \def\Px{1.8}
            \def\Qx{6.8}

            \draw[->,thick] (-0.5,0) -- (10,0) node[below]{$x$};
            \draw[->,thick] (0,-0.5) -- (0,2) node[left]{$y$};
            \draw[blue,very thick,samples=100,smooth,variable=\x,domain=0:5,dashed] plot(\x,{1.25*exp(-(\x-2)^2/0.5)});
            \draw[blue,very thick,samples=100,smooth,variable=\x,domain=0:9.5] plot(\x,{1.25*exp(-(\x-7)^2/0.5)});

            \draw[thick,red,->] (2,1.5) -- node[above,midway]{$\vec{v}$} (3,1.5); 
            \draw[thick,red,->] (7,1.5) -- node[above,midway]{$\vec{v}$} (8,1.5); 
            
            \coordinate (P) at (\Px,{1.25*exp(-(\Px-2)^2/0.5)});
            \coordinate (Px) at (\Px,0);
            
            \fill[black] (P) circle (0.5mm);
            \fill[black] (Px) circle (0.5mm) node[below left]{$x_1$};
            
            \draw[<->] (Px) -- node[right,midway]{$y$} (P);

            \coordinate (Q) at (\Qx,{1.25*exp(-(\Qx-7)^2/0.5)});
            \coordinate (Qx) at (\Qx,0);
            
            \fill[black] (Q) circle (0.5mm);
            \fill[black] (Qx) circle (0.5mm) node[below right]{$x_2$};
            
            \draw[<->] (Qx) -- node[right,midway]{$y$} (Q);

            \draw[<->] (\Px,-0.5) -- node[fill=white]{$v \, \Delta t$} (\Qx,-0.5);
            \draw (\Px,-0.1) -- (\Px,-0.7);
            \draw (\Qx,-0.1) -- (\Qx,-0.7);
        \end{tikzpicture}
    \end{figure}
     
\end{frame}

\begin{frame}{Descripción matemática de las ondas}

    Puede demostrarse fácilmente que la condición $ y(x_2,t_2) = y(x_1,t_1) $ queda satisfecha si la perturbación está dada por $y(x,t) = f(x-s(t))$.

    \vs

    En conclusión, si la magnitud $\xi$ representa una desviación de cierto estado de equilibrio, entonces la propagación de esta perturbación se describe matemáticamente con una función $\xi = f(\vec{r}-\vec{s}(t))$, donde $\vec{r}$ es el vector de posición del punto genérico en el que se evalúa la perturbación y $\vec{s}(t)$ es una función vectorial lineal en el tiempo que describe el desplazamiento de la perturbación. 
    
    \vs

    Toda función que represente una onda se la suele llamar \emph{función de onda}.
     
\end{frame}

\begin{frame}{Descripción matemática de las ondas}

    En virtud de que la función de onda depende tanto de las coordenadas espaciales como del tiempo, podemos analizar la propagación de la perturbación asociada en el tiempo de forma análoga a como lo hicimos para el espacio.

    \vs

    Si ahora $y = f(x_0,t)$ representa la forma del pulso en función del tiempo, esto es cómo cambia la altura $y$ de un punto de la cuerda cuya abscisa es $x_0$ en el tiempo, entonces un \emph{desplazamiento} en el tiempo de esta función está dado por $y = f(t - \tau(x))$ donde, $$ \tau (x) = t_0 + \frac{x-x_0}{v} $$

    \begin{figure}
        \centering
        \begin{tikzpicture}[scale=1]
            \def\Px{1.8}
            \def\Qx{6.8}

            \draw[->,thick] (-0.5,0) -- (10,0) node[below]{$t$};
            \draw[->,thick] (0,-0.5) -- (0,2) node[left]{$y$};
            \draw[blue,very thick,samples=100,smooth,variable=\x,domain=0:5,dashed] plot(\x,{1.25*exp(-(\x-2)^2/0.5)});
            \draw[blue,very thick,samples=100,smooth,variable=\x,domain=0:9.5] plot(\x,{1.25*exp(-(\x-7)^2/0.5)});

            \draw[thick,red,->] (2,1.5) -- node[above,midway]{$\vec{v}$} (3,1.5); 
            \draw[thick,red,->] (7,1.5) -- node[above,midway]{$\vec{v}$} (8,1.5); 
            
            \coordinate (P) at (\Px,{1.25*exp(-(\Px-2)^2/0.5)});
            \coordinate (Px) at (\Px,0);
            
            \fill[black] (P) circle (0.5mm);
            \fill[black] (Px) circle (0.5mm) node[below left]{$t_1$};
            
            \draw[<->] (Px) -- node[right,midway]{$y$} (P);

            \coordinate (Q) at (\Qx,{1.25*exp(-(\Qx-7)^2/0.5)});
            \coordinate (Qx) at (\Qx,0);
            
            \fill[black] (Q) circle (0.5mm);
            \fill[black] (Qx) circle (0.5mm) node[below right]{$t_2$};
            
            \draw[<->] (Qx) -- node[right,midway]{$y$} (Q);

            \draw[<->] (\Px,-0.5) -- node[fill=white]{$\frac{\Delta x}{v}$} (\Qx,-0.5);
            \draw (\Px,-0.1) -- (\Px,-0.7);
            \draw (\Qx,-0.1) -- (\Qx,-0.7);
        \end{tikzpicture}
    \end{figure}

\end{frame}

\begin{frame}{Ondas armónicas}

    En consecuencia, podemos pensar a las ondas como perturbaciones que se propagan en espacio y tiempo.

    \vs

    Un claro ejemplo del desplazamiento en el tiempo de una función es el de una onda armónica. Supongamos que la coordenada $y$ de un punto de abscisa $x_0$ varía en el tiempo según $$ y = f(t) = A \sen \left(\omega t\right), $$ donde $A$ es la amplitud y $\omega$ es la frecuencia angular. En otras palabras, el punto considerado describe un movimiento armónico simple en la dirección vertical.

    \begin{figure}
        \centering
        \begin{tikzpicture}[scale=1]
            \def\A{1}
            \def\L{5}

            \draw[dashed] (0,\A) -- (6.8,\A);
            \draw[dashed] (0,-\A) -- (6.8,-\A);

            \draw[->,thick] (-0.5,0) -- (7,0) node[below]{$t$};
            \draw[->,thick] (0,-1.5) -- (0,1.5) node[left]{$y$};
            \draw[blue,very thick,samples=100,smooth,variable=\x,domain=0:{2*pi}] plot(\x,{\A*sin(\x*360/\L)}) node[above]{$f(t)$};

            \fill[black] (0,\A) circle (0.5mm) node[left]{$+A$};
            \fill[black] (0,-\A) circle (0.5mm) node[left]{$-A$};
            \fill[black] (\L,0) circle (0.5mm) node[below right]{$P$};
            
        \end{tikzpicture}
    \end{figure}

    
\end{frame}

\begin{frame}{Ondas armónicas}

    Si ahora consideramos que se trata de una perturbación armónica que se propaga por la cuerda, debemos incluir en la expresión de la función que describe el movimiento vertical del punto el desplazamiento temporal: $$ y(x,t) = f(t-\tau(x)) = A \sen \left(\omega \left[t - \tau(x)\right]\right) $$ Reemplazando la expresión de $\tau (x)$, se obtiene: $$ y(x,t) = A \sen \left(\omega \left[\left(t - t_0\right) - \frac{\left(x-x_0\right)}{v}\right]\right). $$ Expresión que también podemos reescribir como: $$ y(x,t) = A \sen \left(\omega \, \Delta t - k \, \Delta x\right), $$ donde $k = \dfrac{\omega}{v}$ se conoce como el \emph{número de onda} y el término $\omega \, \Delta t - k \, \Delta x $ es la \emph{fase}, siendo $\Delta t = t - t_0$ y $\Delta x = x - x_0$.

        
\end{frame}

\begin{frame}{Ondas armónicas}

    Como habíamos visto, toda función de onda cumple con la propiedad de que $y(x_1, t_1) = y(x_2, t_2)$, donde $x_2 = x_1 + v \left(t_2-t_1\right)$. Dado que en el caso de las ondas armónicas $y(x,t)$ es una función periódica, tanto en el tiempo como en el espacio, esta propiedad se satisface cuando $\Delta t = t_2 - t_1 = P$, siendo $P$ el periodo de la función. Cuando esto ocurre, el desplazamiento espacial de la onda es $\Delta x = x_2 - x_1 = \lambda$, donde $\lambda$ se conoce como \emph{longitud de onda}. Esto es: $$ \lambda = v \, P$$ Pero como $P = \dfrac{1}{f}$, se obtiene que $$ \lambda = \frac{v}{f}, \qquad \text{ o bien, } \qquad v = \lambda \, f $$ En virtud de que $\omega = 2 \, \pi \, f$, podemos expresar el número de onda en función de la longitud de onda: $$ k = \frac{\omega}{v} = \frac{2 \, \pi \, f}{\lambda \, f}. \quad \text{En consecuencia:} \quad k = \frac{2 \, \pi}{\lambda}. $$ 

            
\end{frame}

\begin{frame}{Ondas armónicas}

    \begin{figure}
        \centering
        \begin{tikzpicture}[scale=1]
            \def\A{1}
            \def\L{5}
            \def\xa{\L/8}
            \def\xb{9*\L/8}

            \coordinate (P) at (\xa,{\A*sin(\xa*360/\L)});
            \coordinate (Q) at (\xb,{\A*sin(\xb*360/\L)});

            \draw[densely dotted] (0,{\A*sin(\xa*360/\L)}) -- (Q);
            \draw[densely dotted] (\xa,0) -- (P);
            \draw[densely dotted] (\xb,0) -- (Q);

            \draw[dashed] (0,\A) -- (6.8,\A);
            \draw[dashed] (0,-\A) -- (6.8,-\A);

            \draw[->,thick] (-0.5,0) -- (7,0) node[below]{$x$};
            \draw[->,thick] (0,-1.5) -- (0,1.5) node[left]{$y$};
            \draw[blue,very thick,samples=100,smooth,variable=\x,domain=0:{2*pi}] plot(\x,{\A*sin(\x*360/\L)});

            \fill[black] (0,\A) circle (0.5mm) node[left]{$+A$};
            \fill[black] (0,-\A) circle (0.5mm) node[left]{$-A$};
            
            \fill[black] (\xa,0) circle (0.5mm) node[below]{$x_1$};
            \fill[black] (\xb,0) circle (0.5mm) node[below]{$x_2$};

            \fill[black] (P) circle (0.5mm);
            \fill[black] (Q) circle (0.5mm);

            \draw[<->] (\xa,1.25) -- node[fill=white]{\small $\lambda$} (\xb,1.25);
            \draw (\xa,1.05) -- (\xa,1.35);
            \draw (\xb,1.05) -- (\xb,1.35);
            
        \end{tikzpicture}
    \end{figure}

    \begin{figure}
        \centering
        \begin{tikzpicture}[scale=1]
            \def\A{1}
            \def\L{5}
            \def\xa{\L/8}
            \def\xb{9*\L/8}

            \coordinate (P) at (\xa,{\A*sin(\xa*360/\L)});
            \coordinate (Q) at (\xb,{\A*sin(\xb*360/\L)});

            \draw[densely dotted] (0,{\A*sin(\xa*360/\L)}) -- (Q);
            \draw[densely dotted] (\xa,0) -- (P);
            \draw[densely dotted] (\xb,0) -- (Q);

            \draw[dashed] (0,\A) -- (6.8,\A);
            \draw[dashed] (0,-\A) -- (6.8,-\A);

            \draw[->,thick] (-0.5,0) -- (7,0) node[below]{$t$};
            \draw[->,thick] (0,-1.5) -- (0,1.5) node[left]{$y$};
            \draw[blue,very thick,samples=100,smooth,variable=\x,domain=0:{2*pi}] plot(\x,{\A*sin(\x*360/\L)});

            \fill[black] (0,\A) circle (0.5mm) node[left]{$+A$};
            \fill[black] (0,-\A) circle (0.5mm) node[left]{$-A$};
            
            \fill[black] (\xa,0) circle (0.5mm) node[below]{$t_1$};
            \fill[black] (\xb,0) circle (0.5mm) node[below]{$t_2$};

            \fill[black] (P) circle (0.5mm);
            \fill[black] (Q) circle (0.5mm);

            \draw[<->] (\xa,1.25) -- node[fill=white]{\small $P$} (\xb,1.25);
            \draw (\xa,1.05) -- (\xa,1.35);
            \draw (\xb,1.05) -- (\xb,1.35);
            
        \end{tikzpicture}
    \end{figure}

\end{frame}

\section{La ecuación de ondas clásica}

\begin{frame}{La ecuación de ondas clásica}

    La ecuación de ondas clásica, o de D'Alembert, es una de las más importantes de la Física dado que describe \emph{cualquier} tipo de onda clásica, independientemente del medio y de la manera en que se propague.

    \vs

    Para derivarla, vamos a considerar nuevamente una cuerda tensa capaz de deformarse, por la que se propaga una onda. Nuevamente, el equilibrio queda caracterizado por la ecuación $y=0$, por lo que la coordenada $y$ representa el apartamiento del equilibrio y, tal como habíamos visto, es un función de la posición $x$ y del tiempo $t$.
    
    \begin{figure}
        \centering
        \begin{tikzpicture}[scale=1]
            \def\Px{1.8}

            \draw[->,thick] (-0.5,0) -- (5,0) node[below]{$x$};
            \draw[->,thick] (0,-0.5) -- (0,2) node[left]{$y$};
            \draw[blue,very thick,samples=100,smooth,variable=\x,domain=0:4.5] plot(\x,{1.25*exp(-(\x-2)^2/0.5)});

            \draw[thick,red,->] (2,1.5) -- node[above,midway]{$\vec{v}$} (3,1.5); 
            
            \coordinate (P) at (\Px,{1.25*exp(-(\Px-2)^2/0.5)});
            \coordinate (Px) at (\Px,0);
            
            \fill[black] (P) circle (0.5mm);
            \fill[black] (Px) circle (0.5mm);
            
            \draw[<->] (Px) -- node[right,midway]{$y$} (P);
        \end{tikzpicture}
    \end{figure}

\end{frame}

\begin{frame}{La ecuación de ondas clásica}

    Consideremos un elemento infinitesimal de la cuerda de longitud $\Delta s$ y masa $\Delta m$, cuya sección transversal tiene un área $A$.

    \begin{figure}
        \centering
        \begin{tikzpicture}[
            declare function ={
                cuerda(\x) = -0.1*(\x-7)^2+4;
            },
            declare function ={
                dcuerda(\x) = -0.2*(\x-7);
                fuerza(\x,\xo) = dcuerda(\xo)*(\x-\xo) + cuerda(\xo);
            }
            ]
            \def\xa{3}
            \def\xb{4.5}
            \def\dx{2}
            \def\dy{0.1}
            \coordinate (P) at (\xa,{cuerda(\xa)});
            \coordinate (Px) at (\xa,0);
            \coordinate (Q) at (\xb,{cuerda(\xb)});
            \coordinate (Qx) at (\xb,0);

            \draw[->,thick] (-0.5,0) -- (8,0) node[below]{\small $x$};
            \draw[->,thick] (0,-0.5) -- (0,5) node[left]{\small $y$};

            \fill[black] (\xa,0) circle (0.5mm) node[below]{\small $x$};
            \fill[black] (\xb,0) circle (0.5mm) node[below]{\small $x + \Delta x$};

            \draw[densely dotted] (Px) -- (P);
            \draw[densely dotted] (Qx) -- (Q);

            \begin{scope}[shift={(P)}]
                \draw (0,0) -- node[pos=0.7,below=-2pt]{\scriptsize $\beta_1$} (-1,0);
                \draw (-0.5,0) arc (180:{180+atan(dcuerda(\xa))}:0.5);
            \end{scope}
            \begin{scope}[shift={(Q)}]
                \draw (0,0) -- node[pos=0.8,above=-3pt]{\scriptsize $\beta_2$} (1,0);
                \draw (0.5,0) arc (0:{atan(dcuerda(\xb))}:0.5);
            \end{scope}

            \draw[black,samples=100,smooth,variable=\x,domain=\xa:\xb,<->] plot(\x,{cuerda(\x)+0.5});
            \node[anchor=south east] at ({0.5*(\xa+\xb)},{cuerda(0.5*(\xa+\xb))+0.5}) {\small $\Delta s$};
            \draw (\xa,{cuerda(\xa)+0.2}) -- (\xa,{cuerda(\xa)+0.6});
            \draw (\xb,{cuerda(\xb)+0.2}) -- (\xb,{cuerda(\xb)+0.6});

            \draw[black,thick,samples=100,smooth,variable=\x,domain=1:7] plot(\x,{cuerda(\x)}) node[right]{\small $y(x,t)$};
            \draw[red,very thick,->] (\xb,{cuerda(\xb)}) -- ({\xb+\dx},{fuerza(\xb+\dx,\xb)}) node[right]{\small $\vec{F}_2$};
            \draw[red,very thick,->] (\xa,{cuerda(\xa)}) -- ({\xa-\dx},{fuerza(\xa-\dx,\xa)}) node[left]{\small $\vec{F}_1$};
            \draw[blue,very thick,samples=100,smooth,variable=\x,domain=\xa:\xb] plot(\x,{cuerda(\x)+\dy});
            \draw[blue,very thick,samples=100,smooth,variable=\x,domain=\xa:\xb] plot(\x,{cuerda(\x)-\dy});
            \draw[blue,very thick] (\xa,{cuerda(\xa)-\dy-0.015}) -- (\xa,{cuerda(\xa)+\dy+0.015});
            \draw[blue,very thick] (\xb,{cuerda(\xb)-\dy-0.015}) -- (\xb,{cuerda(\xb)+\dy+0.015});
            

        \end{tikzpicture}
    \end{figure}

\end{frame}

\begin{frame}{La ecuación de ondas clásica}

    Las fuerzas que actúan sobre el elemento de masa de la cuerda se deben a la interacción con los elementos contiguos. En otras palabras, $\vec{F}_1$ y $\vec{F}_2$ son las fuerzas que los elementos que se encuentran a la izquierda y a la derecha del elemento considerado ejercen sobre este último. Estas fuerzas están dadas por: 
    \begin{align*}
        \vec{F}_1 &= - F_{1,x} \, \ver{e}_x - F_{1,y} \, \ver{e}_y, \\
        \vec{F}_2 &= \phantom{-}F_{2,x} \, \ver{e}_x + F_{2,y}\, \ver{e}_y,
    \end{align*} donde $F_{i,x} = F_i \cos \beta_i$ y $F_{i,y} = F_i \sen \beta_i$, con $i=1,2$, siendo $F_i$ el módulo de cada fuerza. Los ángulos que cada fuerza forma con la dirección horizontal pueden expresarse como: $$ \tan \beta_1 = \frac{F_{1,y}}{F_{1,x}} \qquad \text{y} \qquad \tan \beta_2 = \frac{F_{2,y}}{F_{2,x}}, $$ por lo cual: $F_{1,y} = F_{1,x} \tan \beta_1$ y $F_{2,y} = F_{2,x} \tan \beta_2$.

\end{frame}

\begin{frame}{La ecuación de ondas clásica}

    La resultante de las fuerzas aplicadas es: $$ \vec{R} = \left(-F_{1,x} + F_{2,x}\right) \ver{e}_x + \left(-F_{1,y} + F_{2,y}\right) \ver{e}_y $$ Evidentemente, el elemento de cuerda considerado no está en equilibrio, por lo que, en virtud de la segunda ley de Newton, la resultante debe igularse al producto de la masa del elemento por la aceleración: $$ \left(-F_{1,x} + F_{2,x}\right) \ver{e}_x + \left(-F_{1,y} + F_{2,y}\right) \ver{e}_y = \Delta m \, \vec{a} $$ Ahora bien, tal como se indicó anteriormente, una onda es una perturbación que se propaga pero \emph{sin transporte de materia}. Esto es, la coordenada $x$ de cada punto de la cuerda permanece constante: $x = \text{constante}$ y, en consecuencia, $v_x = 0$ y $a_x = 0$, por lo que la aceleración viene dada por: $$ \vec{a} = 0 \, \ver{e}_x + a_y \, \ver{e}_y.$$

\end{frame}

\begin{frame}{La ecuación de ondas clásica}

    Luego:
    \begin{align*}
        -F_{1,x} + F_{2,x} &= 0, \\
        -F_{1,y} + F_{2,y} &= \Delta m \, a_y.
    \end{align*} De la primera ecuación se obtiene que $F_{1,x} = F_{2,x} = F_x$. La segunda puede entonces reescribirse como: $$ F_x \left(\tan \beta_2 - \tan \beta_1\right) = \Delta m \, a_y. $$ La forma de la onda está descrita por una función $y(x,t)$, puesto que la perturbación, que en este caso corresponde a un desplazamiento vertical, depende tanto del tiempo como del punto de la cuerda. Los ángulos $\beta_1$ y $\beta_2$ dan las direcciones de las fuerzas $\vec{F}_1$ y $\vec{F}_2$, respectivamente, las cuales son tangentes a $y(x,t)$ en los correspondientes puntos $x$ y $x + \Delta x$, por lo que: $$ \tan \beta_1 = \pdiff{y \left(x,t\right)}{x}  \qquad \text{y} \qquad \tan \beta_2 = \pdiff{y \left(x+\Delta x,t\right)}{x}$$
\end{frame}

\begin{frame}{La ecuación de ondas clásica}

    Por otro lado, puesto que $y(x,t)$ da la posición de cada punto de la cuerda en función del tiempo, la componente vertical de la aceleración puede calcularse como $$a_y = \pddiff{y \left(x,t\right)}{t} $$ Además, el diferencial de masa puede escribirse como $\Delta m = \rho \, \Delta V$, donde $\rho$ es la densidad de la cuerda, que se asume constante. El diferencial de volumen puede escribirse como $\Delta V = A \, \Delta s$, donde $A$ es el área de la sección transversal de la cuerda y $\Delta s$ es el elemento de longitud de la cuerda. Si el elemento es suficientemente pequeño, entonces $$\Delta s \approx \sqrt{\left(\Delta x\right)^2 + \left(\Delta y\right)^2} = \Delta x \sqrt{1 + \left(\frac{\Delta y}{\Delta x}\right)^2} $$

\end{frame}

\begin{frame}{La ecuación de ondas clásica}

    Reuniendo los resultados obtenidos, la ecuación de movimiento del elemento de masa de la cuerda queda: $$ F_x \left(\pdiff{y \left(x+\Delta x,t\right)}{x} - \pdiff{y \left(x,t\right)}{x}\right) = \rho \, A \, \Delta x \sqrt{1 + \left(\frac{\Delta y}{\Delta x}\right)^2} \pddiff{y \left(x,t\right)}{t} $$ O bien: $$ F_x \left( \frac{\pdiff{y \left(x+\Delta x,t\right)}{x} - \pdiff{y \left(x,t\right)}{x}}{\Delta x}\right) = \rho \, A \, \sqrt{1 + \left(\frac{\Delta y}{\Delta x}\right)^2} \pddiff{y \left(x,t\right)}{t} $$ Cuando $\Delta x \to 0$, por un lado: $$ \frac{\pdiff{y \left(x+\Delta x,t\right)}{x} - \pdiff{y \left(x,t\right)}{x}}{\Delta x} \to \pddiff{y \left(x,t\right)}{x}, $$ y, por otro lado, $$ \frac{\Delta y}{\Delta x} \to \pdiff{y}{x}.$$

\end{frame}

\begin{frame}{La ecuación de ondas clásica}

    Si se considera que la deformación es pequeña, entonces $$ \pdiff{y}{x} \ll 1 $$ y, en consecuencia, se puede despreciar el segundo término dentro de la raíz cuadrada frente a la unidad: $$ \sqrt{1 + \left(\frac{\Delta y}{\Delta x}\right)^2} \approx 1 .$$ Finalmente, se obtiene: $$ F_x \pddiff{y \left(x,t\right)}{x} = \rho \, A \pddiff{y \left(x,t\right)}{t} $$ 

\end{frame}

\begin{frame}{La ecuación de ondas clásica}

    O bien: $$ \pddiff{y \left(x,t\right)}{t} = \frac{F_x}{\rho \, A} \pddiff{y \left(x,t\right)}{x} $$ Analicemos esta ecuación. Por un lado, como la densidad es la masa $m$ de la cuerda dividida por su volumen $V = A \, L$, donde $A$ es el área de la sección transversal y $L$ es su longitud, el factor $\rho A$ es la densidad lineal de masa: $$ \rho \, A = \frac{m}{V} A = \frac{m}{L},$$ es decir, la masa por unidad de volumen.

    \vs 

    Otro aspecto interesante es la unidad del factor que multiplica la derivada segunda de $y(x,t)$ respecto de $x$ dos veces: $$ \left[\frac{F_x}{\rho \, A}\right] = \left[\frac{F_x \, L}{m}\right] = \frac{\text{N} \, \un{m}}{\un{kg}} = \frac{\un{kg} \, \un{m}}{\un{s}^2} \frac{\un{m}}{\un{kg}} = \frac{\un{m}^2}{\un{s}^2}$$ 

\end{frame}

\begin{frame}{La ecuación de ondas clásica}

    Es decir, dicho factor tiene unidades de rapidez al cuadrado, ¿tendrá que ver con la velocidad de propagación de la onda? Para comprobarlo, analicemos la forma particular de la función de onda.

    \vs 

    Como habíamos visto, toda onda viene descrita por una función del tipo $y(x,t) = f(x-s(t))$, donde $s(t) = x_0 + v \left(t-t_0\right)$. Consideremos, sin pérdida de generalidad, que $y(x,t)$ representa una onda que se propaga hacia la derecha, para el cual $s(0) = 0$, por lo que $x_0 = 0$ y $t_0 = 0$. De esta forma, $y(x,t) = f(x-v \, t)$. Si ponemos $u = x-v \, t$, entonces:
    \begin{columns}
        \begin{column}{0.45\textwidth}
            \begin{align*}
                \pdiff{y(x,t)}{x} &= \fdiff{f}{u} \pdiff{u}{x} = \fdiff{f}{u} \\
                \pddiff{y(x,t)}{x} &= \fddiff{f}{u} \pdiff{u}{x} = \fddiff{f}{u}.
            \end{align*}
        \end{column}
        ~
        \begin{column}{0.45\textwidth}
            \begin{align*}
                \pdiff{y(x,t)}{t} &= \fdiff{f}{u} \pdiff{u}{t} = \fdiff{f}{u} \left(-v\right) \\
                \pddiff{y(x,t)}{t} &= \fddiff{f}{u} \pdiff{u}{t} = \fddiff{f}{u} \, v^2.
            \end{align*}        
        \end{column}
    \end{columns}

\end{frame}

\begin{frame}{La ecuación de ondas clásica}
    
    Es decir: $$ \pddiff{y (x,t)}{t} = v^2 \pddiff{y (x,t)}{x}. $$ Si comparamos esta expresión con la obtenida al analizar la ecuación de movimiento de un elemento de masa de la cuerda, vemos que, efectivamente, $$ v^2 = \frac{F_x}{\rho \, A} $$ Esto es, la velocidad de propagación en la cuerda depende de $F_x$, que podemos identificar como la tensión de la cuerda y la densidad lineal de masa, es decir, de la distribución de masa en la cuerda.
   
\end{frame}

\begin{frame}{La ecuación de ondas clásica}
    
    Más aún, el aspecto central es que la función de onda $y(x,t)$ es solución de la ecuación $$ \pddiff{y (x,t)}{t} = v^2 \, \pddiff{y (x,t)}{x}, $$ de lo cual podemos concluir que esta ecuación es la que describe la propagación de una onda de forma arbitraria en la cuerda. Es por este motivo que se la conoce como \emph{ecuación de onda clásica}.

    \vs

    Además de la ecuación de ondas, que relaciona las derivadas segundas de la perturbación respecto al espacio y al tiempo, vamos a transcribir aquí la relación entre las derivadas primeras para futuras referencias:
    $$ \pdiff{y(x,t)}{t} = \left(-v\right) \, \pdiff{y(x,t)}{x} $$
    
\end{frame}

\begin{frame}{La ecuación de ondas clásica}

    Recordemos que $y$ representa la perturbación que en el caso de la cuerda tensa corresponde al apartamiento respecto del equilibrio, esto es, la posición horizontal $y=0$. Si el estado de requilibrio estuviese representado por la ecuación $y=y_0$ (constante), entonces el apartamiento del equilibrio vendría representado por el desplazamiento respecto de este: $\xi(x,t) = y(x,t) - y_0$. Puesto que $y_0$ es constante, y que $\xi(u) = y(u) - y_0$, con $u= x-v\,t$, se obtiene, por un lado, que: $$ \pdiff{\xi (x,t)}{t} = \left(-v\right) \, \pdiff{\xi(x,t)}{x}, $$ y que: $$ \pddiff{\xi (x,t)}{t} = v^2 \, \pddiff{\xi (x,t)}{x}. $$
    
\end{frame}

\section{Energía transportada por una onda}

\begin{frame}{Energía transportada por una onda}

    Tal como señalamos anteriormente, no hay transporte de materia en la propagación de una onda, sino solamente hay transporte de energía. 
    
    \vs 
    
    Podemos preguntarnos qué tipo de energía es la que se propaga. Si consideramos nuevamente el caso de una onda que se propaga en una cuerda, podemos aprovechar nuestros conocimientos de la mecánica de una partícula y aplicarlos al estudio de un elemento infinitesimal de la cuerda, tal como hicimos para derivar la ecuación de ondas.

\end{frame}

\begin{frame}{Energía transportada por una onda}

    Por un lado, tenemos energía cinética debido al movimiento vertical del elemento de masa. Así, dicho elemento infinitesimal aporta un diferencial de energía cinética dado por $$ \diff E_\text{c} = \frac{1}{2} \diff m \, v_y^2. $$ Pero como $\diff m = \rho \, \diff V $ y $v_y = \pdiff{y}{t}$, entonces: $$ \diff E_\text{c} = \frac{1}{2} \rho \, \diff V \left(\pdiff{y}{t}\right)^2 $$ Resulta conveniente definir la densidad de energía como $$ \mathcal{U} = \fdiff{E}{V}, $$ esto es, la cantidad de energía por unidad de volumen. En este caso, la densidad de energía cinética es $$ \mathcal{U}_\text{c} = \frac{1}{2} \rho \left(\pdiff{y}{t}\right)^2 $$

\end{frame}

\begin{frame}{Energía transportada por una onda}

    Por otro lado, cada elemento de la cuerda almacena energía potencial elástica debido a la deformación que sufre cuando la perturbación pasa por su posición. Podemos calcular el incremento infinitesimal de energía potencial elástica ($\diff E_\text{pe}$) como el trabajo realizado por la fuerza que tensa la cuerda: $$\diff W = \vec{F} \cdot \diff \vec{r}.$$ Si se considera un elemento de volumen de la cuerda suficientemente pequeño, el desplazamiento es igual al alargamiento del elemento de volumen, por lo tanto: $$ \Delta W = F_x \left(\Delta s - \Delta x\right)$$ Además, tenemos nuevamente que: $$ \Delta s \approx \sqrt{\left(\Delta x\right)^2 + \left(\Delta y\right)^2} = \Delta x \sqrt{1 + \left(\frac{\Delta y}{\Delta x}\right)^2}$$

\end{frame}

\begin{frame}{Energía transportada por una onda}

    Bajo la hipótesis planteada, y asumiendo que $$ \frac{\Delta y}{\Delta x} \ll 1, $$ esto es, la deformación relativa o unitaria es pequeña, podemos reemplazar la raíz cuadrada por su expansión en serie de Taylor: $$ \sqrt{1 + \left(\frac{\Delta y}{\Delta x}\right)^2} = 1 + \frac{1}{2} \left(\frac{\Delta y}{\Delta x}\right)^2 + \ldots$$ Si nos quedamos con los dos primeros términos, se obtiene: $$  \Delta s \approx \Delta x \left[1 + \frac{1}{2} \left(\frac{\Delta y}{\Delta x}\right)^2\right] $$

\end{frame}

\begin{frame}{Energía transportada por una onda}

    Luego: $$ \Delta W = F_x \left(\Delta x \left[1 + \frac{1}{2} \left(\frac{\Delta y}{\Delta x}\right)^2\right] - \Delta x\right) $$ Esto es: $$ \Delta W = \frac{1}{2} F_x \, \Delta x \left(\frac{\Delta y}{\Delta x}\right)^2 $$ En el límite, cuando $\Delta x \to 0 $: $$ \diff W = \frac{1}{2} F_x \left(\pdiff{y}{x}\right)^2 \diff x = \diff E_\text{pe} $$

\end{frame}

\begin{frame}{Energía transportada por una onda}

    Para obtener la expresión de la densidad de energía potencial elástica, podemos multiplicar y dividir la expresión anterior por el área de la sección transversal de la cuerda $A$: $$ \diff E_\text{pe} = \frac{1}{2} \frac{F_x}{A} \left(\pdiff{y}{x}\right)^2 A \, \diff x. $$ Pero, $\diff V = A \, \diff x$ y $$ v^2 = \frac{F_x}{\rho \, A}, \qquad \text{por lo que } \qquad \frac{F_x}{A} = \rho \, v^2 $$ y, en consecuencia, $$ \mathcal{U}_\text{pe} = \fdiff{E_\text{pe}}{V} = \frac{1}{2} \rho \, v^2 \left(\pdiff{y}{x}\right)^2. $$

\end{frame}

\begin{frame}{Energía transportada por una onda}

    Ahora bien, la expresión enterior puede reescribirse como: $$ \mathcal{U}_\text{pe} = \fdiff{E_\text{pe}}{V} = \frac{1}{2} \rho \left( v \, \pdiff{y}{x}\right)^2. $$ En virtud de la relación entre las derivadas primeras de la perturbación con respecto al tiempo y al espacio: $$ \pdiff{y(x,t)}{t} = \left(-v\right) \, \pdiff{y(x,t)}{x}, $$ se obtiene: $$ \mathcal{U}_\text{pe} = \fdiff{E_\text{pe}}{V} = \frac{1}{2} \rho \left( - \pdiff{y}{t}\right)^2 = \frac{1}{2} \rho \left(\pdiff{y}{t}\right)^2 = \mathcal{U}_\text{c} $$ En consecuencia, la densidad de energía total transportada por la onda es: $$ \mathcal{U} = \mathcal{U}_\text{c} + \mathcal{U}_\text{ep} = 2 \, \mathcal{U}_\text{c} = 2 \, \mathcal{U}_\text{ep} = \rho \left(\pdiff{y}{t}\right)^2. $$

\end{frame}

\begin{frame}{Potencia}

    Hasta acá vimos la cantidad de energía que transporta una onda. A continuación vamos a analizar qué tan rápido se transmite esa energía. Para ello vamos a calcular la potencia desarrollada por las fuerzas que actúan sobre el elemento de masa de la cuerda tensa considerada anteriormente. Estas fuerzas vienen dadas por:
    \begin{align*}
        \vec{F}_1 &= - F_{1,x} \, \ver{e}_x - F_{1,y} \, \ver{e}_y, \\
        \vec{F}_2 &= \phantom{-}F_{2,x} \, \ver{e}_x + F_{2,y}\, \ver{e}_y,
    \end{align*} donde, nuevamente, $F_{1,y} = F_{1,x} \tan \beta_1$ y $F_{2,y} = F_{2,x} \tan \beta_2$. A su vez, $$ \tan \beta_1 = \pdiff{y \left(x,t\right)}{x}  \qquad \text{y} \qquad \tan \beta_2 = \pdiff{y \left(x+\Delta x,t\right)}{x}.$$ Habíamos visto que en virtud de que el desplazamiento horizontal del elemento es nulo: $F_{1,x} = F_{2,x} = F_x $.
    
\end{frame}

\begin{frame}{Potencia}

    En consecuencia, las fuerzas quedan expresadas de la siguiente manera:
    \begin{align*}
        \vec{F}_1 &= - F_{x} \, \ver{e}_x - F_{x} \pdiff{y \left(x,t\right)}{x} \, \ver{e}_y, \\
        \vec{F}_2 &= \phantom{-}F_{x} \, \ver{e}_x + F_{x} \pdiff{y \left(x+\Delta x,t\right)}{x} \ver{e}_y,
    \end{align*} Por otro lado, la velocidad del elemento de masa viene dada por: $$ \vec{v} = 0 \, \ver{e}_x + \pdiff{y}{t} \, \ver{e}_y. $$ En términos generales, la potencia puede calcularse como $$\fdiff{E}{t} = \vec{F} \cdot \vec{v} $$
    
\end{frame}

\begin{frame}{Potencia}

    Por lo tanto:
    \begin{align*}
        \left(\fdiff{E}{t}\right)_1 &= \fdiff{E}{t} \left(x,t\right) = \vec{F}_1 \cdot \vec{v} = - F_{x} \pdiff{y}{x} \pdiff{y}{t},\\
        \left(\fdiff{E}{t}\right)_2 &= \fdiff{E}{t} \left(x+\Delta x,t\right) = \vec{F}_2 \cdot \vec{v} = F_{x} \pdiff{y}{x} \pdiff{y}{t}.
    \end{align*} Ahora bien, la fuerza $\vec{F}_1$ es la fuerza que el elemento contiguo a a la izquierda del considerado ejerce sobre este, mientras que la fuerza $\vec{F}_2$ es la fuerza que el elemento contigua a la derecha del considerado ejerce sobre este último. Luego, $\left(\fdiff{E}{t}\right)_1$ es la energía por unidad de tiempo que se transmite desde el elemento contiguo a la izquierda hacia el elemento considerado o, la energía por unidad que \emph{entra} en este último. Por su parte, $\left(\fdiff{E}{t}\right)_2$ es la energía por unidad de tiempo que \emph{sale} del elemento considerado.
    
\end{frame}

\begin{frame}{Potencia}

    En virtud de la tercera ley ...

\end{frame}

\end{document}

\begin{frame}{}



\end{frame}