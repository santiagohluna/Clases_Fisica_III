\documentclass[addpoints]{exam}
\usepackage[utf8]{inputenc}
\usepackage[spanish]{babel}
\usepackage[T1]{fontenc}
\usepackage{charter}
\usepackage{amsmath}
\usepackage{amsfonts}
\usepackage{amssymb}
\usepackage{graphicx}
\usepackage{tikz}
\usepackage[outline]{contour} % glow around text
\usetikzlibrary{babel,calc,patterns,decorations.pathmorphing,decorations.markings,arrows.meta,shapes.geometric}
\usetikzlibrary{calc}
\tikzset{>=latex}
\contourlength{1.1pt}
\usepackage{tikz-3dplot}
\usepackage{multicol}
\usepackage{exam-randomizechoices}
\usepackage[left=1cm,right=1cm,top=2cm,bottom=2cm]{geometry}
\usepackage[font=small,labelfont={small,bf},margin=0.5cm,justification=justified]{caption}
\usepackage[font=small,labelfont={small,bf}]{subcaption}
\usepackage[italic,defaultmathsizes]{mathastext}
\usepackage{hyperref}
\usepackage{calculator}
\usepackage[breakable]{tcolorbox}
\usepackage{multirow}
\usepackage{tabularx}
\usepackage{cancel}
\usepackage{tipa}
\usepackage{enumerate}

%\pointpoints{punto}{puntos}
%\bonuspointpoints{punto extra}{puntos extra}

\renewcommand{\solutiontitle}{\textbf{Solución: }}
\renewcommand{\thequestion}{\bfseries\arabic{question}}

\newcommand{\sgn}{\mathop{\mathrm{sgn}}}
\newcommand{\diff}[0]{\mathrm{d}}
\newcommand{\fdiff}[2]{\frac{\mathrm{d} #1}{\mathrm{d} #2}}
\newcommand{\pdiff}[2]{\frac{\partial #1}{\partial #2}}
\newcommand{\fddiff}[2]{\frac{\mathrm{d^2} #1}{\mathrm{d} #2^2}}
\newcommand{\pddiff}[2]{\frac{\partial^2 #1}{\partial {#2}^2}}
\newcommand{\grado}[0]{^{\circ}}
\newcommand{\angulo}[3]{#1\grado \, #2' \, #3''}
\newcommand{\chel}[4]{^{#1}_{#2}\mbox{#3}^{#4}}
\newcommand{\valmed}[1]{\left\langle #1 \right\rangle}
\newcommand{\E}[1]{\times 10^{#1}}
\newcommand{\ver}[1]{\hat{\mathbf{#1}}}
\newcommand{\vecg}[1]{\boldsymbol{#1}}
\newcommand{\iu}{\mathrm{i}}
\newcommand{\norm}[1]{\left\vert\left\vert #1 \right\vert\right\vert}
\newcommand{\abs}[1]{\left\vert #1 \right\vert}
\newcommand{\tens}[1]{\mathbb{#1}}
\newcommand{\rr}{\mathbb{R}}
\newcommand{\un}[1]{\text{#1}}
\newcommand{\logoUNAHUR}{\includegraphics[scale=0.35]{/home/shluna/Proyectos/Clases_Fisica_III/imgs/logo_unahur.png }}
\renewcommand{\arraystretch}{1.5}
\newcommand{\rta}{\textbf{Respuesta: }}
\newcommand{\rtas}{\textbf{Respuestas: }}
\newcommand{\ang}{110}
\newcommand{\angu}{-30}
\newcommand{\rad}{4}
\newcommand{\mg}{1}
\newcommand{\muc}{0.5}
\newcommand{\arc}[1]{{%
  \setbox9=\hbox{#1}%
  \ooalign{\resizebox{\wd9}{\height}{\texttoptiebar{\phantom{A}}}\cr#1}}}

  \colorlet{mydarkblue}{blue!40!black}
  \colorlet{myblue}{blue!30}
  \colorlet{myred}{red!65!black}
  \colorlet{vcol}{green!45!black}
  \colorlet{watercol}{blue!80!cyan!10!white}
  \colorlet{darkwatercol}{blue!80!cyan!80!black!30!white}
  \tikzstyle{water}=[draw=mydarkblue,top color=watercol!90,bottom color=watercol!90!black,middle color=watercol!50,shading angle=0]
  \tikzstyle{horizontal water}=[water,
    top color=watercol!90!black!90,bottom color=watercol!90!black!90,middle color=watercol!80,shading angle=0]
  \tikzstyle{dark water}=[draw=blue!20!black,top color=darkwatercol,bottom color=darkwatercol!80!black,middle color=darkwatercol!40,shading angle=0]
  \tikzstyle{vvec}=[->,very thick,vcol,line cap=round]
  \tikzstyle{force}=[->,myred,very thick,line cap=round]
  \tikzstyle{width}=[{Latex[length=3,width=3]}-{Latex[length=3,width=3]}]

\hypersetup{
%      draft,
   linktocpage=true,
    colorlinks=true,
    linkcolor=blue,
    citecolor=blue,
    filecolor=blue,      
    urlcolor=blue
}

\printanswers
\qformat{\textbf{Ejercicio \thequestion}\hfill}

\pagestyle{headandfoot}
\firstpageheader{Instituto de Tecnología e Ingeniería}{\logoUNAHUR}{Física III}
\firstpageheadrule
\runningheader{Primer parcial}{\logoUNAHUR}{Física III}
\runningheadrule
\firstpagefooter{}{Página \thepage\ de \numpages}{}
\firstpagefootrule
\runningfooter{}{Página \thepage\ de \numpages}{}
\runningfootrule

\begin{document}

\renewcommand{\tablename}{Tabla}

\tdplotsetmaincoords{70}{110}

\begin{tcolorbox}[colback=white,arc=0mm,colframe=black]
    \begin{center}
        \Large\textbf{Física III -- Primer parcial}
    \end{center}
\end{tcolorbox}

\vspace{11pt}

\begin{questions}

    \question Un móvil recorre una pista circular peraltada cierto angulo $\varphi$, tal como se muestra en la Figura~\ref{fig:peralte}, la cual tiene un radio $R$. \\
    \begin{minipage}[c]{0.6\textwidth}
        \begin{parts}
            \part Demostrar que la rapidez con la que el móvil debe recorrer la pista para describir un movimiento circular uniforme debe ser $$ v = \sqrt{g \, R \, \tan \varphi}$$
            \part Demostrar que el periodo del movimiento viene dado por $$ P = 2 \, \pi \sqrt{\frac{R}{g \tan \varphi}} $$
            \part ¿Cuál es la fuerza que obliga al móvil a describir la trayectoria circular?
            \part ¿Es constante la velocidad del móvil? Justifique su respuesta.
        \end{parts}
    \end{minipage}
    \hfill
    \begin{minipage}[c]{0.3\textwidth}
        \begin{center}
            \begin{tikzpicture}[scale=1]
                \node[inner sep=0pt] at (0,0) {\includegraphics[width=\textwidth]{/home/shluna/Proyectos/Clases_Fisica/figs/peralte.pdf}};
                \node at (-0.25,-0.3) {$R$};
                \node at (-0.4,2.25) {$\vec{F}_\text{N}$};
                \node at (-1.3,-2.5) {$\vec{F}_\text{g}$};
                \node at (-1.05,1.2) {$\varphi$};
                \node at (-2.5,0.3) {$\varphi$};
            \end{tikzpicture}
            \captionof{figure}{ }
            \label{fig:peralte}
        \end{center}
    \end{minipage}
    
    \question \label{ej:bloque_plano_MCU} Un bloque de masa $m$ está atado a un clavo situado en el centro de una superficie horizontal sin rozamiento. Se desprecia también el efecto de cualquier otra fuerza externa, como la resistencia del aire. El hilo que une el bloque con el clavo está totalmente estirado y tiene una longitud $R$. El bloque se encuentra inicialmente en reposo, pero en cierto instante se aplica sobre el bloque una fuerza de módulo constante ($F$) y cuya dirección es siempre tangente a la trayectoria, tal como se muestra en la Figura~\ref{fig:plano_bloque_MCU}. En consecuencia, esta fuerza viene dada por $\vec{F} = F \, \ver{e}_\theta $. \label{ej:plano_bloque_MCU}
    
    \begin{minipage}[c]{0.4\textwidth}
        \begin{parts}
            \part Si el módulo de la tensión del hilo puede alcanzar un valor máximo $F_{T,\text{max}}$ sin romperse, demostrar que el tiempo que transcurre desde que el bloque empieza a moverse hasta que el módulo de la tensión alcanza dicho valor máximo viene dado por $$ \Delta t = \frac{1}{F} \sqrt{m \, F_{T,\text{max}} \, R} $$
            \part Mostrar también que el desplazamiento angular del bloque en el intervalo de tiempo calculado en el apartado anterior es $$ \Delta \theta = \frac{F_{T,\text{max}}}{2 \, F}$$
        \end{parts}
    \end{minipage}
    \hfill
    \begin{minipage}[c]{0.55\textwidth}
        \begin{center}
            \begin{tikzpicture}[scale=1,tdplot_main_coords]
                \draw[thick] (-4,-4,0) -- (4,-4,0) -- (4,4,0) -- (-4,4,0) -- cycle;
                \draw[dashed] (0,0,0) circle (3);
                \draw[fill=gray!40] (0,0,0) circle (0.1);
                \draw[fill=gray!40] ({0.1*cos(\ang)},{0.1*sin(\ang)},0) -- ({0.1*cos(\ang)},{0.1*sin(\ang)},0.5) -- ({-0.1*cos(\ang)},{-0.1*sin(\ang)},0.5) -- ({-0.1*cos(\ang)},{-0.1*sin(\ang)},0);
                \draw[fill=gray!40] (0,0,0.5) circle (0.1);
                \draw[thick] (0,-0.1,0.25) arc (-90:90:0.1);
                \draw[thick] (0,0.1,0.25) -- node[anchor=south]{$R$} (0,3,0.25);
                %\draw[thick,-Kite] (4,3,0.25) -- (3.6,3,0.25);
                %\draw[thick,red,-latex] (0,3,0.25) -- node[anchor=north west]{$\vec{v}_{\text{B}2}$} (-3,3,0.25);
                \draw[thick,-latex] (0,3,0.25) -- (-3,3,0.25) node[anchor=north west]{$\vec{F}$};
                % \draw[densely dashed] (-3,3,0.25) -- (-5,3,0.25);
                % \draw[thick,-Kite] (-5,3,0.25) -- (-5.25,3,0.25);
                % \draw[thick,-latex,blue] (-4,3.5,0.25) -- node[anchor=north west]{$\vec{v}_{\text{A}2}$} (-5.5,3.5,0.25);
                \draw[fill=gray!40] (-0.5,2.75,0) -- (-0.5,3.25,0) -- (0.5,3.25,0) -- (0.5,2.75,0) -- cycle;
                \draw[fill=gray!40] (-0.5,2.75,0.5) -- (-0.5,3.25,0.5) -- (0.5,3.25,0.5) -- (0.5,2.75,0.5) -- cycle;
                \draw[fill=gray!40] (0.5,2.75,0) -- (0.5,2.75,0.5) -- (0.5,3.25,0.5) -- (0.5,3.25,0) -- cycle;
                \draw[fill=gray!40] (0.5,3.25,0) -- (0.5,3.25,0.5) -- (-0.5,3.25,0.5) -- (-0.5,3.25,0) -- cycle;
                %\draw[densely dashed] (3.6,3,0.25) -- (0.5,3,0.25);
            \end{tikzpicture}
            \captionof{figure}{ }
            \label{fig:plano_bloque_MCU}
        \end{center}
    \end{minipage}

    \question Dos resortes, de constantes $k_1$ y $k_2$ se someten a la misma fuerza $\vec{F}$ y se observa que sus deformaciones son $\Delta L_1$ y $\Delta L_2$, respectivamente. Demostrar que si $\Delta L_1 < \Delta L_2$, entonces se cumple que $P_1 < P_2$, donde $P_1$ es el periodo de oscilación del resorte de constante $k_1$ y $P_2$ es el periodo del resorte de constante $k_2$.

    \question \label{ej:equilbrio_CR} Una barra rígida que tiene una longitud de $L$ está unida a dos bloques mediante alambres inextensibles y de masa despreciable. El alambre que une la barra con el bloque $A$ pasa por una polea sin rozamiento. Por otro lado, el segmento que va entre el extremo derecho de la barra y la polea forma un ángulo $\psi$ con la horizontal. El bloque $A$ tiene una masa $m_A$ y el bloque $B$ tiene una masa $m_B$, mientras que la barra tiene masa $m_p$. Demostrar que para que la barra permanezca en equilibrio estático en la posición que se muestra en la Figura~\ref{fig:barra}, se debe aplicar una fuerza $\vec{F} = F_x \, \ver{e}_x + F_y \, \ver{e}_y$, en el punto $\vec{r}_F = x_F \, \ver{e}_x$, donde
    \begin{align}
        F_x &= - m_A \, g \cos \psi \\
        F_y &= \left(m_B + m_p - m_A \sen \psi\right) g \\
        x_F &= \left(m_B + \frac{m_p}{2}\right) \frac{L}{m_B + m_p - m_A \sen \psi}.
    \end{align}

    \begin{figure}[ht]
        \centering
        \begin{tikzpicture}[scale=1]
            \draw[thick,-latex] (-5,0) -- (2,0) node[anchor=north]{$x$};
            \draw[thick,-latex] (0,-2) -- (0,2) node[anchor=east]{$y$};
            \fill[black] (0,0) circle (0.3mm);
            \draw[thick] (0,0) -- (60:2);
            \draw[thick] ({2*cos(60)+0.2*cos(30)},{2*sin(60)-0.2*sin(30)}) circle (0.2cm);
            \draw[thick] ({2*cos(60)+0.2*cos(30)+0.2},{2*sin(60)-0.2*sin(30)}) -- ({2*cos(60)+0.4*cos(30)},-0.5);
            \fill[black] ({2*cos(60)+0.2*cos(30)},{2*sin(60)-0.2*sin(30)}) circle (0.3mm);
            \draw (0.5,0) arc (0:60:0.5);
            \draw[thick,fill=gray!40!white] ({2*cos(60)+0.4*cos(30)-0.25},-0.5) rectangle ({2*cos(60)+0.4*cos(30)+0.25},-1);
            \node at ({2*cos(60)+0.2*cos(30)+0.2},-0.75) {$A$};
            \node at (30:0.7) {$\psi$};
            \draw[thick] (-4,0) -- (-4,-0.5);
            \draw[thick,fill=gray!40!white] (-4.3,-0.5) rectangle (-3.7,-1.1);
            \node at (-4,-0.8) {$B$};
            \draw[latex-latex] (-4,1) -- node[fill=white]{$L$} (0,1);
            \draw (-4,0.8) -- (-4,1.2);
            %\draw[latex-latex] (-2,0.5) -- node[fill=white]{$\frac{L}{2}$} (0,0.5);
            %\draw (-2,0.4) -- (-2,0.6);
            \draw[latex-latex,red] (-3,-0.5) -- node[fill=white]{$x_F$} (0,-0.5);
            \draw[red] (-3,-0.3) -- (-3,-0.7);
            \draw[fill=gray!40!white,thick] (-4,-0.1) rectangle (0,0.1);
            \begin{scope}[shift={(-3,0)}]
                \draw[thick,-latex,red] (0,0) -- (120:2) node[anchor=east]{$\vec{F}$};
                \fill[red] (0,0) circle (0.3mm);
                \draw[red] (0.5,0) arc (0:120:0.5);
                \node[red] at (60:0.7) {$\varphi$};
                \draw[red] (0,0) -- (0.75,0);
            \end{scope}
            \fill[black] (-2,0) circle (0.4mm) node[anchor=south,yshift=0.3]{\textsc{cm}}; 
        \end{tikzpicture}
        \caption{Esquema del \ref{ej:equilbrio_CR}. Se indica el centro de masa de la barra (\textsc{cm}).}
        \label{fig:barra}
    \end{figure}

    \question Un bloque de masa $m_A$ se encuentra en reposo sobre una superficie horizontal lisa. Una cuerda inextensible atada a este bloque pasa por una polea, de radio $R$ y masa $m_p$, y en el otro extremo de la cuerda cuelga otro bloque cuya masa es $m_B$, tal como se muestra en la Figura~\ref{fig:vinculados2}. Teniendo en cuenta que el momento de inercia de la polea es $ I = \dfrac{1}{2} m_p \, R^2$, demostrar que el módulo de la aceleración del sistema es $$ a = \frac{m_B \, g}{m_A + m_B + \dfrac{m_p}{2}}. $$

    \begin{figure}[h]
        \centering
        \begin{tikzpicture}[scale=1]
            \fill[pattern=north east lines] (-5,-0.25) -- (-0.5,-0.25) -- (-0.5,-3.5) -- (-0.7,-3.5) -- (-0.7,-0.45) -- (-5,-0.45) -- cycle;
            \draw (0,0) circle (0.25cm);
            \draw (-3,0.25) -- (0,0.25);
            \draw (0.25,0) -- (0.25,-2);
            \draw[fill=gray!40] (-3,-0.25) -- (-3,0.75) -- (-4,0.75) -- (-4,-0.25) -- cycle;
            \draw[fill=gray!40] (0,-2) -- (0.5,-2) -- (0.5,-2.5) -- (0,-2.5) -- cycle;
            \draw[thick] (-5,-0.25) -- (-0.5,-0.25) -- (-0.5,-3.5);
            \node at (-3.5,0.25) {$A$};
            \node at (0.25,-2.25) {$B$};
            \draw[thick] (-0.5,-0.25) -- (0,0);
            \draw[fill=black] (0,0) circle (0.5mm);
        \end{tikzpicture}
        \caption{ }
        \label{fig:vinculados2}
    \end{figure}

\end{questions}

\end{document}