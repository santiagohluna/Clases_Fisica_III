\documentclass[addpoints]{exam}
\usepackage[utf8]{inputenc}
\usepackage[spanish]{babel}
\usepackage[T1]{fontenc}
\usepackage{charter}
\usepackage{amsmath}
\usepackage{amsfonts}
\usepackage{amssymb}
\usepackage{graphicx}
\usepackage{tikz}
\usepackage[outline]{contour} % glow around text
\usetikzlibrary{babel,calc,patterns,decorations.pathmorphing,decorations.markings,arrows.meta,shapes.geometric}
\usetikzlibrary{calc}
\tikzset{>=latex}
\contourlength{1.1pt}
\usepackage{tikz-3dplot}
\usepackage{multicol}
\usepackage{exam-randomizechoices}
\usepackage[left=1cm,right=1cm,top=2cm,bottom=2cm]{geometry}
\usepackage[font=small,labelfont={small,bf},margin=0.5cm,justification=justified]{caption}
\usepackage[font=small,labelfont={small,bf}]{subcaption}
\usepackage[italic,defaultmathsizes]{mathastext}
\usepackage{hyperref}
\usepackage{calculator}
\usepackage[breakable]{tcolorbox}
\usepackage{multirow}
\usepackage{tabularx}
\usepackage{cancel}
\usepackage{tipa}
\usepackage{enumerate}

%\pointpoints{punto}{puntos}
%\bonuspointpoints{punto extra}{puntos extra}

\renewcommand{\solutiontitle}{\textbf{Solución: }}
\renewcommand{\thequestion}{\bfseries\arabic{question}}

\newcommand{\sgn}{\mathop{\mathrm{sgn}}}
\newcommand{\diff}[0]{\mathrm{d}}
\newcommand{\fdiff}[2]{\frac{\mathrm{d} #1}{\mathrm{d} #2}}
\newcommand{\pdiff}[2]{\frac{\partial #1}{\partial #2}}
\newcommand{\fddiff}[2]{\frac{\mathrm{d^2} #1}{\mathrm{d} #2^2}}
\newcommand{\pddiff}[2]{\frac{\partial^2 #1}{\partial {#2}^2}}
\newcommand{\grado}[0]{^{\circ}}
\newcommand{\angulo}[3]{#1\grado \, #2' \, #3''}
\newcommand{\chel}[4]{^{#1}_{#2}\mbox{#3}^{#4}}
\newcommand{\valmed}[1]{\left\langle #1 \right\rangle}
\newcommand{\E}[1]{\times 10^{#1}}
\newcommand{\ver}[1]{\hat{\mathbf{#1}}}
\newcommand{\vecg}[1]{\boldsymbol{#1}}
\newcommand{\iu}{\mathrm{i}}
\newcommand{\norm}[1]{\left\vert\left\vert #1 \right\vert\right\vert}
\newcommand{\abs}[1]{\left\vert #1 \right\vert}
\newcommand{\tens}[1]{\mathbb{#1}}
\newcommand{\rr}{\mathbb{R}}
\newcommand{\nn}{\mathbb{N}}
\newcommand{\zz}{\mathbb{Z}}
\newcommand{\un}[1]{\text{#1}}
\newcommand{\logoUNAHUR}{\includegraphics[scale=0.35]{/home/shluna/Proyectos/Clases_Fisica_III/imgs/logo_unahur.png }}
\renewcommand{\arraystretch}{1.5}
\newcommand{\rta}{\textbf{Respuesta: }}
\newcommand{\rtas}{\textbf{Respuestas: }}
\newcommand{\ang}{110}
\newcommand{\angu}{-30}
\newcommand{\rad}{4}
\newcommand{\mg}{1}
\newcommand{\muc}{0.5}
\newcommand{\arc}[1]{{%
  \setbox9=\hbox{#1}%
  \ooalign{\resizebox{\wd9}{\height}{\texttoptiebar{\phantom{A}}}\cr#1}}}

  \colorlet{mydarkblue}{blue!40!black}
  \colorlet{myblue}{blue!30}
  \colorlet{myred}{red!65!black}
  \colorlet{vcol}{green!45!black}
  \colorlet{watercol}{blue!80!cyan!10!white}
  \colorlet{darkwatercol}{blue!80!cyan!80!black!30!white}
  \tikzstyle{water}=[draw=mydarkblue,top color=watercol!90,bottom color=watercol!90!black,middle color=watercol!50,shading angle=0]
  \tikzstyle{horizontal water}=[water,
    top color=watercol!90!black!90,bottom color=watercol!90!black!90,middle color=watercol!80,shading angle=0]
  \tikzstyle{dark water}=[draw=blue!20!black,top color=darkwatercol,bottom color=darkwatercol!80!black,middle color=darkwatercol!40,shading angle=0]
  \tikzstyle{vvec}=[->,very thick,vcol,line cap=round]
  \tikzstyle{force}=[->,myred,very thick,line cap=round]
  \tikzstyle{width}=[{Latex[length=3,width=3]}-{Latex[length=3,width=3]}]

\hypersetup{
%      draft,
   linktocpage=true,
    colorlinks=true,
    linkcolor=blue,
    citecolor=blue,
    filecolor=blue,      
    urlcolor=blue
}

\printanswers
\qformat{\textbf{Ejercicio \thequestion}\hfill}

\pagestyle{headandfoot}
\firstpageheader{Instituto de Tecnología e Ingeniería}{\logoUNAHUR}{Física III}
\firstpageheadrule
\runningheader{Recuperatorio del segundo parcial}{\logoUNAHUR}{Física III}
\runningheadrule
\firstpagefooter{}{Página \thepage\ de \numpages}{}
\firstpagefootrule
\runningfooter{}{Página \thepage\ de \numpages}{}
\runningfootrule

\begin{document}

\renewcommand{\tablename}{Tabla}

\tdplotsetmaincoords{70}{110}

\begin{tcolorbox}[colback=white,arc=0mm,colframe=black]
    \begin{center}
        \Large\textbf{Física III -- Recuperatorio del segundo parcial}
    \end{center}
\end{tcolorbox}

\begin{itemize}
    \item \makebox[0.75\textwidth]{Nombre y apellido:\enspace\hrulefill}
    \item \makebox[0.75\textwidth]{DNI:\enspace\hrulefill}
\end{itemize}

\begin{tcolorbox}[colback=black!5!white,arc=0mm,breakable,pad at break*=1mm,colframe=black!25!white,title=\textbf{\textcolor{black}{Instrucciones generales}}]
    \begin{itemize}
        \item Expresar todas las unidades en el Sistema Internacional.
        \item Tanto las respuestas como los desarrollos correspondientes deben escribirse con bolígrafo.
        \item Todas las hojas a entregar deben estar numeradas y se debe indicar el nombre, apellido y número de DNI en cada una.
        \item Todas las respuestas deben estar correctamente justificadas.
    \end{itemize}
\end{tcolorbox}

\begin{questions}

    \question Demostrar que las funciones que tienen la forma $\psi \left(x,t\right) = f \left(t - \dfrac{x}{v}\right)$ y $\psi \left(x,t\right) = f \left(t + \dfrac{x}{v}\right)$ representan ondas que se propagan a lo largo de la dirección $x$ con rapidez $v$ hacia $+\infty$ y hacia $-\infty$, respectivamente.

    \question Considere la siguiente distribución: $$ y (x) = A \, \exp \left[- \frac{\left(x-x_0\right)^2}{\sigma^2} \right] \cos \left(a \, x\right). $$ 
    \begin{parts}
        \part sasas
    \end{parts}

    \question Una cuerda de masa $m$ y densidad $\rho$ y cuya sección transversal tiene un área $A$, que se dispone en dirección horizontal, pasa por una polea de radio $R$ y se une a un bloque de masa $M$. El otro extremo se une a una pared en el punto $O$ que se encuentra a una distancia $L$ de la polea, medida en dirección horizontal, tal como se muestra en la Figura~\ref{fig:cuerda_bloque_polea}. Si los puntos de la cuerda oscilan en la dirección vertical describiendo un movimiento armónico simple: \label{ej:cuerda_bloque_polea}
    \begin{parts}
        \part ¿Cuáles son las condiciones de borde que debe satisfacer la función que describe la onda que se propague por la cuerda?
        \part ¿Qué tipo de onda pueden propagarse por la cuerda para satisfacer esas condiciones de borde?
        \part Escriba la expresión de la función de onda correspondiente.
        \part Explicar por qué solamente se pueden propagar ondas de frecuencias bien determinadas.
        \part Mostrar que, en este caso, las frecuencias lineales posibles vienen dadas por $$ f_n = \frac{n}{2} \sqrt{\frac{M \, g}{m \, L}}, $$ donde $n \in \nn$. Verificar la unidad de la frecuencia lineal.
    \end{parts}

    \begin{figure}[ht]
        \centering
        \begin{tikzpicture}[scale=1]
            \coordinate (O) at (0,0);
            \coordinate (P) at (6,0);

            \fill[pattern=north east lines] (-0.2,-1) rectangle (0,1);
            
            \draw[ultra thick,brown] (O) -- (P);
            \draw[thick] (0,1) -- (0,-1);
            \fill[black] (O) circle (0.5mm) node[anchor=north west]{$O$};

            \draw[ultra thick,brown] (6.5,-0.5) -- (6.5,-2.5);
            \draw[ultra thick,brown] (P) arc (90:0:0.5);
            \draw[thick,fill=gray!40!white] (6,-2.5) rectangle (7,-3.5);
            \node at (6.5,-3) {$M$};

            \draw[thick,fill=gray!40!white] (6,-0.5) circle (0.5);
            \fill[black] (6,-0.5) circle (0.5mm);

            \draw[latex-latex] (0,0.5) -- node[fill=white]{$L$} (6,0.5);
            \draw (6,0.6) -- (6,0.1);

        \end{tikzpicture}
        \caption{Esquema del ejercicio \ref{ej:cuerda_bloque_polea}.}
        \label{fig:cuerda_bloque_polea}
    \end{figure}

    \pagebreak 

    \question Los campos eléctrico y magnético de una onda electromagnética monocromática plana están polarizados según $\ver{e}_y$ y $\ver{e}_z$, respectivamente. Se sabe además, que la onda electromagnética se propaga de forma transversal. La amplitud del primero es $E_0$.
    \begin{parts}
        \part Escriba las expresiones vectoriales del campo eléctrico y del campo magnético.
        \part Hallar la expresión del vector de Poynting.
        \part Las perturbaciones de los campos eléctrico y magnético, ¿se propagan con la misma rapidez y con la misma frecuencia? Justifique su respuesta.
        \part ¿En qué dirección y sentido se propaga esta onda electromagnética?
    \end{parts}
    

\end{questions}

\end{document}