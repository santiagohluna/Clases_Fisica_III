\documentclass[addpoints]{exam}
\usepackage[utf8]{inputenc}
\usepackage[spanish]{babel}
\usepackage[T1]{fontenc}
\usepackage{charter}
\usepackage{amsmath}
\usepackage{amsfonts}
\usepackage{amssymb}
\usepackage{graphicx}
\usepackage{tikz}
\usepackage[outline]{contour} % glow around text
\usetikzlibrary{babel,calc,patterns,decorations.pathmorphing,decorations.markings,arrows.meta,shapes.geometric}
\usetikzlibrary{calc}
\tikzset{>=latex}
\contourlength{1.1pt}
\usepackage{tikz-3dplot}
\usepackage{multicol}
\usepackage{exam-randomizechoices}
\usepackage[left=1cm,right=1cm,top=2cm,bottom=2cm]{geometry}
\usepackage[font=small,labelfont={small,bf},margin=0.5cm,justification=justified]{caption}
\usepackage[font=small,labelfont={small,bf}]{subcaption}
\usepackage[italic,defaultmathsizes]{mathastext}
\usepackage{hyperref}
\usepackage{calculator}
\usepackage[breakable]{tcolorbox}
\usepackage{multirow}
\usepackage{tabularx}
\usepackage{cancel}
\usepackage{tipa}
\usepackage{enumerate}

%\pointpoints{punto}{puntos}
%\bonuspointpoints{punto extra}{puntos extra}

\renewcommand{\solutiontitle}{\textbf{Solución: }}
\renewcommand{\thequestion}{\bfseries\arabic{question}}

\newcommand{\sgn}{\mathop{\mathrm{sgn}}}
\newcommand{\diff}[0]{\mathrm{d}}
\newcommand{\fdiff}[2]{\frac{\mathrm{d} #1}{\mathrm{d} #2}}
\newcommand{\pdiff}[2]{\frac{\partial #1}{\partial #2}}
\newcommand{\fddiff}[2]{\frac{\mathrm{d^2} #1}{\mathrm{d} #2^2}}
\newcommand{\pddiff}[2]{\frac{\partial^2 #1}{\partial {#2}^2}}
\newcommand{\grado}[0]{^{\circ}}
\newcommand{\angulo}[3]{#1\grado \, #2' \, #3''}
\newcommand{\chel}[4]{^{#1}_{#2}\mbox{#3}^{#4}}
\newcommand{\valmed}[1]{\left\langle #1 \right\rangle}
\newcommand{\E}[1]{\times 10^{#1}}
\newcommand{\ver}[1]{\hat{\mathbf{#1}}}
\newcommand{\vecg}[1]{\boldsymbol{#1}}
\newcommand{\iu}{\mathrm{i}}
\newcommand{\norm}[1]{\left\vert\left\vert #1 \right\vert\right\vert}
\newcommand{\abs}[1]{\left\vert #1 \right\vert}
\newcommand{\tens}[1]{\mathbb{#1}}
\newcommand{\rr}{\mathbb{R}}
\newcommand{\un}[1]{\text{#1}}
\newcommand{\logoUNAHUR}{\includegraphics[scale=0.35]{/home/shluna/Proyectos/Clases_Fisica_III/imgs/logo_unahur.png }}
\renewcommand{\arraystretch}{1.5}
\newcommand{\rta}{\textbf{Respuesta: }}
\newcommand{\rtas}{\textbf{Respuestas: }}
\newcommand{\ang}{110}
\newcommand{\angu}{-30}
\newcommand{\rad}{4}
\newcommand{\mg}{1}
\newcommand{\muc}{0.5}
\newcommand{\arc}[1]{{%
  \setbox9=\hbox{#1}%
  \ooalign{\resizebox{\wd9}{\height}{\texttoptiebar{\phantom{A}}}\cr#1}}}

  \colorlet{mydarkblue}{blue!40!black}
  \colorlet{myblue}{blue!30}
  \colorlet{myred}{red!65!black}
  \colorlet{vcol}{green!45!black}
  \colorlet{watercol}{blue!80!cyan!10!white}
  \colorlet{darkwatercol}{blue!80!cyan!80!black!30!white}
  \tikzstyle{water}=[draw=mydarkblue,top color=watercol!90,bottom color=watercol!90!black,middle color=watercol!50,shading angle=0]
  \tikzstyle{horizontal water}=[water,
    top color=watercol!90!black!90,bottom color=watercol!90!black!90,middle color=watercol!80,shading angle=0]
  \tikzstyle{dark water}=[draw=blue!20!black,top color=darkwatercol,bottom color=darkwatercol!80!black,middle color=darkwatercol!40,shading angle=0]
  \tikzstyle{vvec}=[->,very thick,vcol,line cap=round]
  \tikzstyle{force}=[->,myred,very thick,line cap=round]
  \tikzstyle{width}=[{Latex[length=3,width=3]}-{Latex[length=3,width=3]}]

\hypersetup{
%      draft,
   linktocpage=true,
    colorlinks=true,
    linkcolor=blue,
    citecolor=blue,
    filecolor=blue,      
    urlcolor=blue
}

\printanswers
\qformat{\textbf{Ejercicio \thequestion}\hfill}

\pagestyle{headandfoot}
\firstpageheader{Instituto de Tecnología e Ingeniería}{\logoUNAHUR}{Física III}
\firstpageheadrule
\runningheader{Recuperatorio del primer parcial}{\logoUNAHUR}{Física III}
\runningheadrule
\firstpagefooter{}{Página \thepage\ de \numpages}{}
\firstpagefootrule
\runningfooter{}{Página \thepage\ de \numpages}{}
\runningfootrule

\begin{document}

\renewcommand{\tablename}{Tabla}

\tdplotsetmaincoords{70}{110}

\begin{tcolorbox}[colback=white,arc=0mm,colframe=black]
    \begin{center}
        \Large\textbf{Física III -- Recuperatorio del primer parcial}
    \end{center}
\end{tcolorbox}

\vspace{11pt}

\begin{questions}

    \question Un bloque de masa $m$ está unido a un clavo situado en el centro de una superficie horizontal sin rozamiento mediante un resorte de constante $k$. Se desprecia también el efecto de cualquier otra fuerza externa, como la resistencia del aire, tal como se muestra en la Figura~\ref{fig:plano_bloque_1}. Si el bloque describe un movimiento circular uniforme, mostrar que: \label{ej:plano_bloque_1} 
    \begin{enumerate}[a)]
        \item El radio de la trayectoria circular depende del periodo del movimiento según: $$ R = \frac{k \, L_0}{k - m \, \omega^2} $$ donde $L_0$ es la longitud \emph{natural} del resorte y $\omega = \dfrac{2 \, \pi}{P} $.
        \item La rapidez con la que el bloque recorre la trayectoria circular viene dada por: $$ v = \sqrt{\frac{k}{m} R \left(R - L_0\right)}. $$
    \end{enumerate}

    \begin{figure}[ht]
        \centering
        \begin{tikzpicture}[scale=1,tdplot_main_coords]
            \draw[thick] (-4,-4,0) -- (4,-4,0) -- (4,4,0) -- (-4,4,0) -- cycle;
            \draw[dashed] (0,0,0) circle (3);
            \draw[fill=gray!40] (0,0,0) circle (0.1);
            \draw[fill=gray!40] ({0.1*cos(\ang)},{0.1*sin(\ang)},0) -- ({0.1*cos(\ang)},{0.1*sin(\ang)},0.5) -- ({-0.1*cos(\ang)},{-0.1*sin(\ang)},0.5) -- ({-0.1*cos(\ang)},{-0.1*sin(\ang)},0);
            \draw[fill=gray!40] (0,0,0.5) circle (0.1);
            \draw[thick] (0,-0.1,0.25) arc (-90:90:0.1);
            \draw[decoration={coil,segment length = 1mm,amplitude = 2mm,aspect = 0.3,post length = 3mm,pre length = 3mm},decorate,thick,black!50] (0,0.1,0.25) -- node[above=4pt]{$k$} (0,3,0.25);
            %\draw[thick,-Kite] (4,3,0.25) -- (3.6,3,0.25);
            %\draw[thick,red,-latex] (0,3,0.25) -- node[anchor=north west]{$\vec{v}_{\text{B}2}$} (-3,3,0.25);
            \draw[thick,-latex] (0,3,0.25) -- node[anchor=north west]{$\vec{v}$} (-3,3,0.25);
            % \draw[densely dashed] (-3,3,0.25) -- (-5,3,0.25);
            % \draw[thick,-Kite] (-5,3,0.25) -- (-5.25,3,0.25);
            % \draw[thick,-latex,blue] (-4,3.5,0.25) -- node[anchor=north west]{$\vec{v}_{\text{A}2}$} (-5.5,3.5,0.25);
            \draw[fill=gray!40] (-0.5,2.75,0) -- (-0.5,3.25,0) -- (0.5,3.25,0) -- (0.5,2.75,0) -- cycle;
            \draw[fill=gray!40] (-0.5,2.75,0.5) -- (-0.5,3.25,0.5) -- (0.5,3.25,0.5) -- (0.5,2.75,0.5) -- cycle;
            \draw[fill=gray!40] (0.5,2.75,0) -- (0.5,2.75,0.5) -- (0.5,3.25,0.5) -- (0.5,3.25,0) -- cycle;
            \draw[fill=gray!40] (0.5,3.25,0) -- (0.5,3.25,0.5) -- (-0.5,3.25,0.5) -- (-0.5,3.25,0) -- cycle;
            %\draw[densely dashed] (3.6,3,0.25) -- (0.5,3,0.25);
        \end{tikzpicture}
        \caption{ }
        \label{fig:plano_bloque_1}
    \end{figure}

    \question Un resorte de constante $k$ se sujeta por uno de sus extremos a un techo, mientras que el otro extremo se une a un bloque de masa $m$. Se observa que en el equilibrio, la deformación del resorte es $\Delta L$. Luego, el sistema formado por el resorte y el bloque se coloca sobre una superficie horizontal lisa (se desprecia el rozamiento entre el bloque y la superficie) y se lo pone a oscilar. Demostrar que el periodo del oscilador armónico simple puede calcularse con la expresión: $$ P = 2 \, \pi \sqrt{\frac{\Delta L}{g}} $$

    

\end{questions}

\end{document}