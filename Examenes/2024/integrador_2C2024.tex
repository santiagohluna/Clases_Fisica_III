\documentclass[addpoints]{exam}
\usepackage[utf8]{inputenc}
\usepackage[spanish]{babel}
\usepackage[T1]{fontenc}
\usepackage{charter}
\usepackage{amsmath}
\usepackage{amsfonts}
\usepackage{amssymb}
\usepackage{graphicx}
\usepackage{tikz}
\usepackage[outline]{contour} % glow around text
\usetikzlibrary{babel,calc,patterns,decorations.pathmorphing,decorations.markings,arrows.meta,shapes.geometric}
\usetikzlibrary{calc}
\tikzset{>=latex}
\contourlength{1.1pt}
\usepackage{tikz-3dplot}
\usepackage{multicol}
\usepackage{exam-randomizechoices}
\usepackage[left=1cm,right=1cm,top=2cm,bottom=2cm]{geometry}
\usepackage[font=small,labelfont={small,bf},margin=0.5cm,justification=justified]{caption}
\usepackage[font=small,labelfont={small,bf}]{subcaption}
\usepackage[italic,defaultmathsizes]{mathastext}
\usepackage{hyperref}
\usepackage{calculator}
\usepackage[breakable]{tcolorbox}
\usepackage{multirow}
\usepackage{tabularx}
\usepackage{cancel}
\usepackage{tipa}
\usepackage{enumerate}

%\pointpoints{punto}{puntos}
%\bonuspointpoints{punto extra}{puntos extra}

\renewcommand{\solutiontitle}{\textbf{Solución: }}
\renewcommand{\thequestion}{\bfseries\arabic{question}}

\newcommand{\sgn}{\mathop{\mathrm{sgn}}}
\newcommand{\diff}[0]{\mathrm{d}}
\newcommand{\fdiff}[2]{\frac{\mathrm{d} #1}{\mathrm{d} #2}}
\newcommand{\pdiff}[2]{\frac{\partial #1}{\partial #2}}
\newcommand{\fddiff}[2]{\frac{\mathrm{d^2} #1}{\mathrm{d} #2^2}}
\newcommand{\pddiff}[2]{\frac{\partial^2 #1}{\partial {#2}^2}}
\newcommand{\grado}[0]{^{\circ}}
\newcommand{\angulo}[3]{#1\grado \, #2' \, #3''}
\newcommand{\chel}[4]{^{#1}_{#2}\mbox{#3}^{#4}}
\newcommand{\valmed}[1]{\left\langle #1 \right\rangle}
\newcommand{\E}[1]{\times 10^{#1}}
\newcommand{\ver}[1]{\hat{\mathbf{#1}}}
\newcommand{\vecg}[1]{\boldsymbol{#1}}
\newcommand{\iu}{\mathrm{i}}
\newcommand{\norm}[1]{\left\vert\left\vert #1 \right\vert\right\vert}
\newcommand{\abs}[1]{\left\vert #1 \right\vert}
\newcommand{\tens}[1]{\mathbb{#1}}
\newcommand{\rr}{\mathbb{R}}
\newcommand{\un}[1]{\text{#1}}
\newcommand{\logoUNAHUR}{\includegraphics[scale=0.35]{/home/shluna/Proyectos/Clases_Fisica_III/imgs/logo_unahur.png }}
\renewcommand{\arraystretch}{1.5}
\newcommand{\rta}{\textbf{Respuesta: }}
\newcommand{\rtas}{\textbf{Respuestas: }}
\newcommand{\ang}{110}
\newcommand{\angu}{-30}
\newcommand{\rad}{4}
\newcommand{\mg}{1}
\newcommand{\muc}{0.5}
\newcommand{\arc}[1]{{%
  \setbox9=\hbox{#1}%
  \ooalign{\resizebox{\wd9}{\height}{\texttoptiebar{\phantom{A}}}\cr#1}}}

  \colorlet{mydarkblue}{blue!40!black}
  \colorlet{myblue}{blue!30}
  \colorlet{myred}{red!65!black}
  \colorlet{vcol}{green!45!black}
  \colorlet{watercol}{blue!80!cyan!10!white}
  \colorlet{darkwatercol}{blue!80!cyan!80!black!30!white}
  \tikzstyle{water}=[draw=mydarkblue,top color=watercol!90,bottom color=watercol!90!black,middle color=watercol!50,shading angle=0]
  \tikzstyle{horizontal water}=[water,
    top color=watercol!90!black!90,bottom color=watercol!90!black!90,middle color=watercol!80,shading angle=0]
  \tikzstyle{dark water}=[draw=blue!20!black,top color=darkwatercol,bottom color=darkwatercol!80!black,middle color=darkwatercol!40,shading angle=0]
  \tikzstyle{vvec}=[->,very thick,vcol,line cap=round]
  \tikzstyle{force}=[->,myred,very thick,line cap=round]
  \tikzstyle{width}=[{Latex[length=3,width=3]}-{Latex[length=3,width=3]}]

\hypersetup{
%      draft,
   linktocpage=true,
    colorlinks=true,
    linkcolor=blue,
    citecolor=blue,
    filecolor=blue,      
    urlcolor=blue
}

\printanswers
\qformat{\textbf{Ejercicio \thequestion}\hfill}

\pagestyle{headandfoot}
\firstpageheader{Instituto de Tecnología e Ingeniería}{\logoUNAHUR}{Física III}
\firstpageheadrule
\runningheader{Examen integrador}{\logoUNAHUR}{Física III}
\runningheadrule
\firstpagefooter{}{Página \thepage\ de \numpages}{}
\firstpagefootrule
\runningfooter{}{Página \thepage\ de \numpages}{}
\runningfootrule

\begin{document}

\renewcommand{\tablename}{Tabla}

\tdplotsetmaincoords{70}{110}

\begin{tcolorbox}[colback=white,arc=0mm,colframe=black]
    \begin{center}
        \Large\textbf{Física III -- Examen integrador}
    \end{center}
\end{tcolorbox}

\begin{itemize}
    \item \makebox[0.75\textwidth]{Nombre y apellido:\enspace\hrulefill}
    \item \makebox[0.75\textwidth]{DNI:\enspace\hrulefill}
\end{itemize}

\begin{questions}

    \question Un resorte de constante $k_1$ se une a un bloque de masa $m$. Luego el sistema formado por estos se dispone en dirección horizontal y se lo hace oscilar alrededor de la posición de equilibrio del resorte. Si se observa que el periodo es $P_1$, mostrar que si el resorte se reemplaza por otro de constante $k_2$, el sistema formado por este nuevo resorte y el bloque oscilará con un periodo $P_2$ dado por: $$ P_2 = P_1 \sqrt{\frac{k_1}{k_2}}.$$
    
    \question \label{ej:volante_hueco} Un volante, que tiene forma de un cilindro hueco, de masa $M$, radio interno $R_0$ y radio externo $R$, puede girar alrededor de un eje horizontal que pasa por su centro geométrico, el cual se indica con el punto $O$ de la Figura~\ref{fig:volante_hueco}. Una cuerda arrollada a la circunferencia del volante une a este último con un balde de masa $m$. El volante se considera un cuerpo rígido cuyo momento de inercia es $I = \dfrac{1}{2} M \, \left(R^2 + R_0^2\right)$, mientras que el balde se asume como una masa puntual.
    \begin{parts}
        \part Mostrar que si el rozamiento entre el eje y el volante produce un torque $\vec{T}_\text{roz}$, entonces la aceleración con la que desciende el balde es: $$ \vec{a} = \left(\frac{T_\text{rox}}{R} - m \, g\right) \left(m + \xi \, M \right)^{-1},$$ donde $\xi = \dfrac{I}{M \, R^2}$ es el factor momento de inercia.
        \part Analizar los casos en que $R_0 = 0$ (cilindro macizo) y $R_0 \approx R$ (cilindro hueco de pared delgada). ¿Para cuál de ellos la aceleración es mayor?
        \part ¿Bajo qué condición o condiciones el balde descendería a velocidad constante?
    \end{parts}

    \begin{figure}[ht]
        \centering
        \begin{tikzpicture}
            \def\R{2.5}
            \def\d{4}
            \def\w{0.25}
            \def\l{1}
            \def\L{1}
            \def\h{3}
            \def\x{0.7}

            \coordinate (O) at (0,0);
            \coordinate (A) at (-\R,0);
            \coordinate (D) at (-\R,-\h);

            \draw[thick,fill=gray!40] (O) circle (\R);
            \draw[thick,fill=white] (O) circle ({\x*\R});
            \begin{scope}[rotate=30]
                \draw[ultra thick] (O) -- (0:{\x*\R});
                \draw[ultra thick] (O) -- (120:{\x*\R});
                \draw[ultra thick] (O) -- (240:{\x*\R});
            \end{scope}
            \draw[thick,fill=gray!40] (O) circle ({0.05*\R});

            \draw[-latex] (O) -- node[fill=white]{$R$} (60:\R);
            \draw[-latex] (O) -- node[fill=white]{$R_0$} ({270+45}:{\x*\R});

            \draw[thick,->] ({-(\R+1)},0) -- ({\R+1},0) node[anchor=north]{$x$};
            \draw[thick,->] (0,{-(\R+1)}) -- (0,{\R+1}) node[anchor=east]{$y$};
            \draw[thick] (A) -- (D);
            
            %% Balde.
            \begin{scope}[shift={(D)}]
                \draw[thick] (0.5,-0.5) arc (0:180:0.5);
                \draw[thick,fill=gray!30] (-0.5,-0.5) -- (0.5,-0.5) -- (0.4,-1.3) -- (-0.4,-1.3) -- cycle;
            \end{scope}

            \fill[black] (O) circle (0.5mm) node[anchor=north east]{$O$};
            \fill[black] (A) circle (0.5mm) node[anchor=south east]{$A$};
            \fill[black] (D) circle (0.5mm) node[anchor=south east]{$A'$};

        \end{tikzpicture}
        \caption{Esquema del Ejercicio~\ref{ej:volante_hueco}.}
        \label{fig:volante_hueco}
    \end{figure}

    \pagebreak

    \question Utilizar el principio de Fresnel para deducir que los ángulos de incidencia y de reflexión de un rayo sobre una superficie son iguales.

    \question Deducir la ley de Snell a partir del principio de Huygens y luego utilizarla para demostrar que la desviación que sufre un rayo luminoso al incidir sobre una lámina de caras paralelas de ancho $d$ es: $$ d = .$$

    \question %% Experimento de Young

\end{questions}

\end{document}