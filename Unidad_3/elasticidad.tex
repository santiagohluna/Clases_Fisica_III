%\documentclass[11pt]{beamer}
\documentclass[11pt,handout,aspectratio=1610]{beamer}

\usepackage[utf8]{inputenc}
\usepackage[T1]{fontenc}
\usepackage[spanish]{babel}
\usepackage{latexsym} 
\usepackage{amsmath}
\usepackage{amsfonts}
\usepackage{amssymb}
\usepackage{esint}
\usepackage{array}
\usepackage{multirow}
\usepackage{xcolor}
\usepackage{graphicx}
\usepackage{tikz}
\usepackage{tikz-3dplot}
\usetikzlibrary{babel}
\usetikzlibrary{calc,patterns,decorations.pathmorphing,decorations.markings}
\usepackage{xcolor}
\usepackage{epstopdf}
\usepackage[nointegrals]{wasysym}
\usepackage{hyperref}
\usepackage{cancel}

\usetheme{Berkeley}
\usecolortheme{seahorse}
\uselanguage{Spanish}

\newcommand{\sgn}{\mathop{\text{sgn}}}
\newcommand{\diff}[0]{\text{d}}
\newcommand{\fdiff}[2]{\dfrac{\text{d} #1}{\text{d} #2}}
\newcommand{\pdiff}[2]{\frac{\partial #1}{\partial #2}}
\newcommand{\fddiff}[2]{\frac{\text{d^2} #1}{\text{d} #2^2}}
\newcommand{\grado}[0]{^{\circ}}
\newcommand{\chel}[4]{^{#1}_{#2}\text{#3}^{#4}}
\newcommand{\valmed}[1]{\left\langle #1 \right\rangle}
\newcommand{\E}[1]{\times 10^{#1}}
\newcommand{\ver}[1]{\hat{\vec{#1}}}
\newcommand{\vecg}[1]{\boldsymbol{#1}}
\newcommand{\iu}{\text{i}}
\newcommand{\norm}[1]{\left\vert\left\vert #1 \right\vert\right\vert}
\newcommand{\abs}[1]{\left\vert #1 \right\vert}
\newcommand{\tens}[1]{\mathbb{#1}}
\newcommand{\rr}{\mathbb{R}}
\newcommand{\logoUNAHUR}{\includegraphics[scale=0.15]{/home/shluna/Proyectos/Clases_Fisica_III/imgs/logo-universidad-nacional-de-hurlingham_preview_rev_1.png}}
\newcommand{\vs}{\vspace{11pt}}
\newcommand{\un}[1]{\text{#1}}

\title{Introducción a la teoría de la elasticidad}
\subtitle{Unidad 3}
\author{Física III}
\institute{Instituto de Tecnología e Ingeniería \\ \vspace{0.25cm} Universidad Nacional de Hurlingham}
\date{ }
\logo{\logoUNAHUR}

\AtBeginSection[]{
  \begin{frame}
  \vfill
  \centering
  \begin{beamercolorbox}[sep=8pt,center,shadow=true,rounded=true]{title}
    \usebeamerfont{title}\insertsectionhead\par%
  \end{beamercolorbox}
  \vfill
  \end{frame}
}

\tdplotsetmaincoords{70}{110}

\begin{document}

\frame{\titlepage}

\begin{frame}{En esta clase veremos:}
    \tableofcontents
\end{frame}

\section{Introducción}

\begin{frame}{Introducción}

    Cuando estudiamos el cuerpo rígido, lo definimos como un sistema de partículas indeformable. Sin embargo, la experiencia nos muestra que todos los cuerpos se deforman en mayor o menor medida cuando se aplican fuerzas sobre ellos.

    \vs

    El propósito de esta unidad es estudiar los aspectos básicos de la denominada \emph{Teoría de la elasticidad:}

    \begin{block}{Teoría de la elasticidad}
        Puede decirse que la elasticidad es la rama de la Física que se ocupa de estudiar la deformación de los cuerpos.
    \end{block}

    A continuación, vamos a comenzar el estudio de la elasticidad definiendo la deformación.

\end{frame}

\section{Deformación}

\begin{frame}{Deformación}

    Cuando estudiamos el movimiento del cuerpo rígido, lo definimos como un sistema de partículas cuyas distancias relativas son constantes: $$ \ell_{ij} = \norm{\vec{r}_i - \vec{r}_j} $$ donde $\ell_{ij}$ es la distancia entre las partículas $i$ y $j$ que se encuentran en los puntos $\vec{r}_i$ y $\vec{r}_j$, respectivamente. 

    \vs 

    Por supuesto, si esta condición es válida para cualquier par de puntos del cuerpo rígido, en particular lo es para el punto $i$ y el origen de un sistema de referencia que se encuentra dentro del cuerpo y que se mueve solidariamente junto con este y, por lo tanto: $$ \ell_i = \norm{\vec{r}_i} $$ donde ahora $\ell_i$ es la distancia del punto $i$ al origen de dicho sistema de referencia.

\end{frame}

\section{Tensión y fuerzas corporales}

\begin{frame}{Tensión o esfuerzo}



\end{frame}

\begin{frame}{Fuerzas corporales o de volumen}
    
s

\end{frame}

\section{Ecuaciones constitutivas y módulos elásticos.}

\begin{frame}{Ecuaciones constitutivas}



\end{frame}

\begin{frame}{Módulos elásticos}



\end{frame}

\begin{frame}{Esto es todo por hoy}

    \begin{center}
        {\huge ¡Muchas gracias!}

        \vs

        Ahora a repasar y practicar.
    \end{center}

\end{frame}

\end{document}

\begin{frame}{}



\end{frame}